\documentclass[12pt]{article}

\usepackage[utf8]{inputenc}
\usepackage[swedish]{babel}

\usepackage{tikz}
\usepackage{hyperref}
\usetikzlibrary{shapes,arrows}
\usepackage{fancyhdr}
\usepackage{float}

\usepackage[a4paper, total={14cm,20cm}]{geometry}
\lhead{TSEA83}
\rhead{VT18}
\pagestyle{fancy}


\begin{document}

\begin{center}
\vspace*{15mm}
{\Huge Designspecifikation: FPGA Z80}\\
\vspace{15mm}
{\large 26 februari 2018}\\
\vspace{15mm}
Noah Hellman: noahe116 \\
Dennis Derecichei: dende301 \\
Yousef Hashem: youha847 \\
Jakob Arvidsson: jakar180 \\
\end{center}

\vspace{2cm}
\tableofcontents
\newpage

\section{Analys}
\subsection{CPU}
Vi ska bygga en halvt pipelinad z80 där hämt- och exekveringsfaserna
överlappar. Programminnet och dataminnet är delat och ligger utanför
processorn. Det finns en databuss som är 8 bitar och en addressbuss som är 16
bitar. Det finns även en kontrollbuss på 16 bitar som innehåller styrsignaler
utifrån såsom INT och MEMRQ. Instruktionerna är mellan 8 och 32 bitar långa.
Det finns olika adresseringsmoder såsom
\begin{itemize}
    \item \texttt{ld a,55} - omedelbar, a laddas med 55,
    \item \texttt{jr 5} - relativ, PC adderas med 5,
    \item \texttt{jp 4578} - direkt, PC sätts till 4578,
    \item \texttt{ld a,(b)} - indirekt, a laddas värdet som b pekar på,
    \item \texttt{ld a,(IX+5)} - indexerad, a laddas med addressen IX pekar på
        plus 5.
\end{itemize}
Vi vill kunna ladda program via USB vid start av processorn. CPU:n ska kunna
köra generella Z80-program och utföra alla uppgifter förutom det som
IO-kontrollerna hanterar.

\subsection{TI ASIC}
En speciell krets kommer behövas för att koppla samman processorn med
IO-enheterna som en TI84 förväntas ha. Z80:n använder sig av ett portsystem som
denna krets ska implementera. Varje port är kopplad till en viss enhet och har
ett visst syfte. Till exempel port 10 används för att skicka kommandon till
displayen medan port 11 används för att skicka bilddata till displayen. ASIC:en
förmedlar avbrott mellan CPU:n och IO-kontrollerna, samt dirigerar databussen
till rätt port så processorn kan kommunicera med den specifika IO-enheten.

\subsection{Flashmodul}
Vi vill kunna skicka program som ska köras till processorn via USB. Vi ska då
använda en flashmodul som läser datan från USB och skriver det till minnet,
eventuellt avbryter processorns nuvarande program. Eventuellt kan flashmodulen
använda BUSRQ-signalen för att kunna skriva till minnet direkt (DMA).

\subsection{IO}
IO kopplas till processorn med addressbussen, databussen samt styrsignaler.
CPU:n meddelar att den vill komma åt IO via IORQ-signalen och lägger addressen
till porten på addressbussen. En mux kopplar därefter databussen till rätt
IO-enhet. 

IO-enheterna kommer (precis som TI84:n gör) använda sig av följande portar:
\begin{itemize}
    \item Port 0: Tangentbord
    \item Port 10: LCD kommandon/status
    \item Port 11: LCD data
\end{itemize}

IO-enheterna kan göra interrupts med INT- och NMI-signalerna. Beroende på
vilket mode som är inställt interagerar IO-enheten på olika sätt. Med till
exempel mode 0 kan IO-enheten efter INT-signalen placera en instruktion på
databussen som CPU:n kommer exekvera nästa klockpuls. I mode 2 kan IO-enheten
lägga en index på databussen till en avbrottsrutin som är lagrad i en tabell i
minnet. Processorn kommer då hoppa till den rutinen efter den nuvarande
instruktionen. Den moden som ska användas kan väljas med IM (interrupt mode)
instruktionen. Dessa avbrott kan också stängas av och sättas på med DI och EI
instruktionerna.

\subsection{GPU}
Vi kommer inte behöva någon enhet för att göra grafiska beräkningar, de kommer
utföras av processorn. Vi behöver endast ett videominne som processorn kan
skriva till och en VGA-motor som skriver videominnet till skärmen.

\subsection{Programmering}
Z80:n programmeras med Z80:ns instruktionsset. Z80 är en populär processor med
många verktyg tillgängliga. Vi kommer programmera och använda oss av
tillgängliga assemblers för att skapa maskinkod.

\section{Milstolpe}
Vi har tänkt att göra klart z80:ns instruktioner (ej avbrott) samt koppla den
till en VGA-motor som kan skriva registernas värden till VGA-skärmen.

\section{Blockschema}
\subsection{TI84}
\begin{figure}[H]
    \centering
    \def\svgwidth{0.9\columnwidth}
    \input{ti84.pdf_tex}
\end{figure}
\subsection{Z80}
\begin{figure}[H]
    \centering
    \def\svgwidth{0.87\columnwidth}
    \documentclass[main.tex]{subfiles}

\begin{document}
\subsection{Z80}
Z80:n kan delas upp i tre sektioner; kontroll, register och ALU. De är anslutna
med varandra via databussen, addressbussen och en del tunnlar som alla kan ses
i blockschema \ref{diag:z80}. Kontrollsektionen har även kontrollsignaler till
alla delar av processorn i form av ett långt kontrollord. Kontrollordet kallas
för \mono{cw} och alla dess signaler i finns med i blockschemat förutom
muxadresser till adress- och databussen samt alla lässignaler för alla
register.

\subfile{sections/hw/ctrl}
\subfile{sections/hw/rf}
\subfile{sections/hw/alu}
\subfile{sections/hw/instr.tex}
\newpage

\end{document}

\end{figure}
\subsection{TI ASIC}
\begin{figure}[H]
    \centering
    \def\svgwidth{1.2\columnwidth}
    \input{asic.pdf_tex}
\end{figure}
\end{document}
