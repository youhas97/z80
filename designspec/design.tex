\documentclass[12pt]{article}

\usepackage[utf8]{inputenc}
\usepackage[swedish]{babel}

\usepackage{tikz}
\usepackage{hyperref}
\usetikzlibrary{shapes,arrows}
\usepackage{fancyhdr}
\usepackage{float}

\usepackage[a4paper, total={14cm,20cm}]{geometry}
\lhead{TSEA83}
\rhead{VT18}
\pagestyle{fancy}


\begin{document}

\begin{center}
\vspace*{2cm}
{\Huge Designspecifikation: FPGA Z80}\\
\vspace{2cm}
{\large 26 februari 2018}\\
\vspace{2cm}
Noah Hellman: noahe116 \\
Dennis Derecichei: dende301 \\
Yousef Hashem: youha847 \\
Jakob Arvidsson: jakar180 \\
\end{center}

\vspace{2cm}
\tableofcontents
\newpage

\section{Inledning}
\section{Blockschema}
\subsection{Z80}
\subsection{TI84}
\section{Analys}
\subsection{CPU}
Vi ska bygga en halvt pipelinad z80 där fetch och exec överlappar. Progamminnet
och dataminnet är delat och ligger utanför processorn. Det finns en databuss
som är 8 bitar och en addressbuss som är 16 bitar. Instruktionerna är mellan 8
och 32 långa. Det finns olika adresseringsmoder såsom
\begin{itemize}
    \item \texttt{ld a,55} - omedelbar, a laddas med 55,
    \item \texttt{jr 5} - relativ, PC adderas med 5,
    \item \texttt{jp 4578} - direkt, PC sätts till 4578,
    \item \texttt{ld a,(b)} - indirekt, a laddas värdet som b pekar på,
    \item \texttt{ld a,(IX+5)} - indexerad, a laddas med addressen IX pekar på
        plus 5.
\end{itemize}
Vi vill kunna ladda programmen via USB vid start av processorn. CPU:n ska kunna
köra generella z80-program och utföra alla uppgifter förutom det som
IO-controllers hanterar.

\subsection{IO}
\subsubsection{Åtkomst till IO från CPU}
IO kopplas till processorn med addressbussen, databussen samt kontrollsignaler.
CPU:n meddelar att den vill komma åt IO via IO request-signalen (IORQ) och
lägger addressen till porten på addressbussen. En mux kopplar därefter
databussen till rätt IO-enhet. 

\subsubsection{Interrupts från IO}
IO-enheterna kan göra interrupts med INT- och NMI-signalerna. Beroende på
vilket mode som är inställt interagerar IO-enheten på olika sätt. Med till
exempel mode 0 kan IO-enheten efter INT-signalen placera en instruktion på
databussen som CPU:n kommer exekvera nästa klockpuls. I mode 2 kan IO-enheten
lägga en index på databussen till en interruptrutin som är lagrad i en tabell i
minnet.  Processorn kommer då hoppa till den rutinen efter den nuvarande
instruktionen.  Det mode som ska användas kan väljas med IM (interrupt mode)
instruktionen.  Dessa interrupts kan också stängas av och sättas på med DI och
EI instruktionerna.

\subsection{GPU}
Vi kommer inte behöva någon enhet för att göra grafiska beräkningar, de kommer
utföras av processorn. Vi behöver endast ett videominne som processorn kan
skriva till och en VGA-motor som skriver videominnet till skärmen.

\subsection{Programmering}
Z80:n programmeras med Z80:ns instruktionsset. Z80 är en populär processor med
många verktyg tillgängliga. Vi kommer programmera och använda tillgängliga
assemblers för att skapa maskinkod.

\section{Milstolpe}
Vi har tänkt att göra klart z80:ns instruktioner (inte interrupts) samt koppla
den till en VGA-motor som kan skriva registernas värden till VGA-skärmen.
\end{document}
