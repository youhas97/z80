\documentclass[12pt]{article}

% encoding
\usepackage[utf8]{inputenc}
\usepackage[swedish]{babel}

% bibliography
\usepackage[backend=bibtex8]{biblatex}
\addbibresource{references.bib}
\usepackage{csquotes}

% equations
\usepackage{amsmath}

% split files
\usepackage{subfiles}

% table util, multicolumn etc
\usepackage{array}
\newcolumntype{C}[1]{>{\centering\arraybackslash}p{#1}}

% place figures here (H)
\usepackage{float}

% subfigures with captions
\usepackage{subcaption}

% titleformat for paragraphs
\usepackage{titlesec}

% hyperlinks for references, urls
\usepackage{hyperref}

% page dimensions
\usepackage[a4paper, total={16cm, 25cm}]{geometry}

% landscape appendices
\usepackage[paper=portrait,pagesize]{typearea}

% labeling
\usepackage{scrextend}

% include graphics
\usepackage{graphicx}

% wrap figures in text
\usepackage{wrapfig}

% captions on side of figure
\usepackage{sidecap}

% command listings
\usepackage{listings}
\lstset{basicstyle=\footnotesize\ttfamily}

% vhdl listings
\usepackage[newfloat]{minted}
\usepackage{caption}

% use colors
\usepackage{xcolor}

% SI
\usepackage{siunitx}

% diagrams
\usepackage{tikz}
\usepackage{tikz-timing}
% make arrows larger, IDA missing this package
%\usetikzlibrary{arrows.meta}
\usetikzlibrary{arrows,shapes,automata,petri,positioning,calc}

% monospace font
\newcommand\mono[1]{\texttt{#1}}

% header for sections under subsubsections
\titleformat{\paragraph}
{\normalfont\normalsize\bfseries}{\theparagraph}{1em}{}
\titlespacing*{\paragraph}
{0pt}{3.25ex plus 1ex minus .2ex}{1.5ex plus .2ex}


\begin{document}

\begin{center}
\vspace*{2cm}
{\Huge Designspecifikation: FPGA Z80}\\
\vspace{2cm}
{\large 26 februari 2018}\\
\vspace{2cm}
Noah Hellman: noahe116 \\
Dennis Derecichei: dende301 \\
Yousef Hashem: youha847 \\
Jakob Arvidsson: jakar180 \\
\end{center}

\vspace{2cm}
\tableofcontents
\newpage

\section{Inledning}
\section{Analys}
\subsection{CPU}
Vi ska bygga en halvt pipelinad z80 där fetch och exec överlappar. Progamminnet
och dataminnet är delat och ligger utanför processorn. Det finns en databuss
som är 8 bitar och en addressbuss som är 16 bitar. Instruktionerna är mellan 8
och 32 långa. Det finns olika adresseringsmoder såsom
\begin{itemize}
    \item \texttt{ld a,55} - omedelbar, a laddas med 55,
    \item \texttt{ld a,b} - direkt, a laddas värdet i b,
    \item \texttt{ld a,(b)} - indirekt, a laddas värdet som b pekar på,
    \item \texttt{ld a,(55)} - indirekt, a laddas värdet på address 55,
    \item \texttt{ld a,(IX+5)} - indexerad, a laddas med addressen IX pekar på
        plus 5.
\end{itemize}
Vi vill kunna ladda programmen via USB vid start av processorn. CPU:n ska kunna
köra generella z80-program och utföra alla uppgifter förutom det som
IO-controllers hanterar.

\subsection{GPU/IO}
IO kopplas till processorn med addressbuss och en IO request-signal (IORQ).
CPU:n meddelar att den vill komma åt IO via IORQ och lägger addressen till
porten på addressbussen. En mux kopplar därefter databussen till rätt IO-enhet.
IO-enheterna kan skicka bus request-signal (BUSRQ) så att processorn sätter
bussen till high impedance så att IO-enheten kan överföra data mellan IO och
minnet.

\section{Blockschema}
\section{Milstolpe}

\end{document}
