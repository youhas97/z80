\documentclass[main.tex]{subfiles}

\begin{document}
\subsection{Operationdekodare}
Operationdekodaren var svår att implementera på ett fint och lättläst sätt,
och har därför implementerats som en enda jättestor case-when som samspelar med
en liten case-when. Poängen med den här case-satsen är att kodningen på
operationerna kan behållas och användas som olika fall i case-satsen. Den lilla
case-satsen är till för att hålla koll på vilket prefix instruktionen är på för
tillfället, för att kunna lista ut instruktion som ska köras härnäst. Eftersom
instruktionerna skiljer från prefix till prefix så var den här implementationen
det smidigaste sättet att lista ut vilken instruktion som ska köras från en
sträng av bitar.

Varje fall i case-satsen gör ett funktionsanrop till en funktion som gör det
instruktionen ska göra. I de flesta funktionerna görs ett internt
funktionsanrop beroende på vad som ska göras (läsa från minnet, skriva till
minnet etc). Anledningen det interna funktionsanropet görs är för att samma kod
inte ska behöva skrivas i alla instruktioner som gör samma sak (läser från minnet,
skriver till minnet etc).

\end{document}
