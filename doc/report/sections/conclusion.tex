\documentclass[main.tex]{subfiles}

\begin{document}
\newpage
\section{Slutsats}
Projektets mål uppnåddes; en replikation av en Z80-processor har byggdes och
användes för att bygga en TI-miniräknare. 

\subsection{Slutmål}
Slutmålet var att bygga en TI-84 Plus men en TI-83p visade sig vara väldigt lik
utifrån men ha färre saker som behövde implementeras inuti. TI-84p har till
exempel en dedikerad komponent för att räkna ut MD5-hashes som TI-83p hanterar
med mjukvara. TI-84p har även en andra uppsättning av timers som går efter en
kristall och kan justeras mer utförligt av programmeraren. Jämfört med TI-83p
har TI-84p dessutom ett väldigt stort antal portar som gruppen var osäkra på
vilka som var nödvändiga och om några kunde anta konstanta värden. I efterhand
verkar beslutet att istället göra en TI-83p bra eftersom det sparade mycket
tid. Med den tid som projektet gick var det dessutom svårt att hinna med och få
TI-83p att fungera helt.

\subsection{Tillvägagång}
I början av projektet skedde allt arbete med simulering. Det första som antogs
var processorns ALU och registerfil för att därefter kunna testa utföra enkla
instruktioner. Mycket tid gick åt till att göra ALU:n i början, med tiden blev
den enklare och enklare. Något som har varit till oerhört stor nytta vid
designen av ALU:n är enhetstester. För varje instruktion som implementerades
skrevs ett antal enhetstester i en testbänk efter manualen och emulatorernas
beteende. På så sätt kunde ALU:n justeras till att testerna gick igenom och
instruktionen var rätt implementerad. Senare vid varje ändring kördes alla
tester och om något gick fel visade testet vilken instruktion, med vilka
operander och vad som gick fel, resultatet eller någon flagga. Utan dessa test
hade det varit väldigt lätt att misstag vid ändringar gick oupptäckta tills
långt senare.

När ALU:n och registerfilen fungerade påbörjades instruktionsdekodaren och de
andra delarna av processorn. Ett simulerat minne byggdes också för att lagra
program. Då kunde processorn börja köra enkla program som lästes från minnet.
Därefter utökades processorn med fler instruktioner och testning med FPGA:n
påbörjades. I det här stadiet var enhetstester också oerhört användbara. Nu
skrevs tester med Z80-assembly kördes på processorn. Vid en implementation av
en instruktion skrevs ett program i en textfil som använde instruktionen och
testade resultatet. Därefter användes ett skript för att assembla programmet,
simulera projektet med GHDL och ladda programmet från filen till det simulerade
minnet. Då kunde vågorna från simulationen ses med hjälp av GTKWave.
Programmen som skrevs utförde flera tester i rad, om något test misslyckades
hoppade programmet till en rutin som lagrade en felkod i ackumulatorn och
halt:ade processorn. Om alla test gick igenom lagrades en annan kod och
processorn halt:ade. På så sätt kunde ett stort test köras för att testa flera
instruktioner i taget för att snabbt hitta misstag vid ändringar.

Det här var smidigt för att testa fel på individuella instruktioner, men det
var svårt att använda för att felsöka problem som uppstod när operativsystemet
och andra program så småningom testades. Dessa kör miljontals instruktioner och
använder mycket loopar vilket inte går att simulera inom en rimlig tid. Dessa
var tvungna att köras på FGPA:n. För att kunna felsöka program på FPGA:n
skapades en display på skärmen som visade värdet av alla register. Dessa gör
inte så stor nytta när processorn kör i \SI{6}{\mega\hertz} så brytpunkter och
instruktionsstegning implementerades också. På så sätt gick det relativt fort
att stega igenom programmet tills något gick fel. Fel hittades genom att
jämföra resultatet med en emulator. När man gick för långt kunde man spara
adressen, sätta en brytpunkt och starta om. Det här ledde till att rätt många
fel med processorn upptäcktes och fixades. Vissa problem uppstod dock rätt
långt in i programmet och var inte alls lätta att hitta snabbt på det här
sättet.

Vid ett tillfälle var det tydligt att något gick fel eftersom processorn
hamnade på en adress som emulatorn aldrig kom till. Då implementerades ett sätt
att spåra alla hopp som utförs av processorn. Spåraren aktiverades och det
felaktiga programmet kördes. Processorn kom till fel ställe och nådde en
\mono{halt}. Då sparades alla hopp till en fil och med binärsökning kunde det
exakta stället där processorn kommit fel hittas snabbt. Denna metod gjorde att
felsökningen gick oerhört fort i slutet av projektet och alla kvarstående fel
löstes under ett par dagar.

\subsection{Utökningar}
Många av TI-miniräknarna använder sig av Z80-processorn men med små skillnader
på räknarna som också skulle kunna implementeras. De delar som är gemensamma
kan separareras från de komponenter som skiljer. För varje miniräknare kan
dessa komponenter implementeras. En av dessa miniräknare som kan implementeras
är förstås TI-84p.

De resterande delarna av TI-83p:an skulle också kunna implementeras.
Exekveringsmasken och ROM-skrivskyddet skulle tillåta exekvering av illvilliga
eller felaktiga program utan att behöva återställa ROM. Länkporten skulle
dessutom kunna implementeras, till exempel med UART-porten. Vissa utvecklare
har använt länkporten för att implementera flerspelarlägen för sina spel. Om
två FPGA:s kopplades samman skulle två personer kunna spela tillsammans med var
sin dedikerad skärm och tangentbord. Länkporten skulle dessutom kunna användas
för att ladda program och annan data till och från en dator.

Z80-processorn skulle dessutom kunna användas för att bygga andra datorer som
inte är TI-miniräknare. Processorn är fullt kompatibel med Intel 8080. Datorer
som använder den skulle också kunna byggas. Exempelvis skulle en dator som kan
köra CP/M-operativsystemet kunna byggas.

En annan möjlig utökning är att emulera miniräknarens LCD-skärm bättre. Vår
VGA-skärm har nu endast två olika färger men LCD-skärmens pixlar har olika
intensitet. När en pixel aktiveras går den kontinuerligt från släckt till tänd
och inte instant som i vår implementation. Det här används av många spel för
att skapa gråskala genom att snabbt sätta på och stänga av pixlar. Det här
skulle kunna implementeras genom att skapa ett andra bildminne som lagrar varje
pixels intensitet. Med en viss frekvens kan pixlars intensitet inkrementeras
eller dekrementeras beroende på om pixeln håller på att släckas eller tändas.

\end{document}
