\documentclass[main.tex]{subfiles}

\begin{document}
\section{Hårdvara}
Konstruktionen är uppdelad i tre delar; processorn, TI-ASIC:en och kontroller
för externa enheter. Dess sammansättning kan ses i blockschema \ref{diag:comp}.
I blockschemat syns även att det finns olika klockor som genereras. Processorn
kan köra på olika frekvenser medan ASIC:en alltid kör på 33MHz och VGA-motorn
på 25MHz. Resterande komponenter använder systemklockan på 100MHz. Klockorna
kan även stängas av vid bland annat brytpunkter som använts för debugging.

Det finns en gemensam 8-bitars databuss som processorn, ASIC:en och även minnet
delar på. Kontrollbussen och adressbussen från processorn går direkt till
ASIC:en som hanterar dem först innan något går till de externa kontrollerna.
Adressen behöver till exempel översättas innan den når det fysiska minnet. I
ASIC:en sker även hanteringen av alla de olika portarna i miniräknaren. ASIC:en
talar i sin tur med de externa komponenterna och skickar indata via portarna.

\subfile{sections/hw/z80.tex}
\subfile{sections/hw/ti.tex}
\subfile{sections/hw/ext.tex}

\end{document}
