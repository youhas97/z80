\documentclass[main.tex]{subfiles}

\begin{document}
\subsection{ALU:n}
 ALU:n tar in två operander som kallas för op1 samt op2. Op1 lagras i en
 temporär ackumulator ACT och op2 lagras i ett temporärt register TMP.
 Instruktionen som ska utföras på operanderna kommer direkt från IR och heter
 op (operation). I ALU:n finns det 4 räknare som bestämmer vilket resultat som
 ska användas utifrån vilken instruktion (op) som tas in.Flaggorna 
 S,Z,F5,H,F3,P/V,N och C bestäms utifrån respektive instruktion. När en
 instruktion ska användas muxas därför flaggorna utifrån om flaggan ska sättas
 till 1,0 eller om flaggans tidigare värde ska behållas. Om flaggans värde ska
 behållas i en instruktion kommer flaggans tidigare värde att kopieras till det
 nuvarande.
\end{document}
