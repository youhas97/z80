\documentclass[main.tex]{subfiles}

\begin{document}
\newpage
\subsubsection{ALU-sektion}
ALU-sektionen hanterar matematiska beräkningar samt hantering av flaggor.
Registerna \mono{ACT} och \mono{TMP}, ALU:n själv, en mux till \mono{P}-flaggan
samt en komparator för addressindatan tillhör ALU-sektionen. Hur de är
sammankopplade visas i mitten av blockschema \ref{diag:z80}.

\paragraph{\mono{ACT} och \mono{TMP}}
Det primära syftet av registerna \mono{ACT} och \mono{TMP} att lagra
operanderna under ALU:ns operation så att resultatet direkt kan läggas på
databussen. De två operanderna till ALU:n måste alltid laddas till \mono{ACT}
och \mono{TMP}. \mono{ACT} kallas för den temporära ackumulatorn och kan endast
ladda värden från \mono{A} i registerfilen. Vissa instruktioner använder endast
en operand, i så fall laddas den till \mono{TMP} och \mono{ACT}:s värde har
ingen påverkan på resultatet. \mono{TMP} kan ladda ett godtyckligt värde
eftersom den är ansluten till databussen. \mono{TMP} kan även skriva till
databussen och kan användas för temporär lagring, därav namnets ursprung.

\paragraph{Flaggor}
Flaggorna lagras i registerfilen men dess beräkning sker i ALU-sektionen. Vid
en ALU-instruktion får ALU:n alltid in de tidigare flaggorna direkt från
\mono{F} i registerfilen. Utifrån de tidigare flaggorna och resultatet beräknas
de nya flaggorna inom ALU:n. Under klockintervallet som ALU:n används läggs de
nya flaggorna på indatan till \mono{F} och lässignalen för \mono{F} går aktiv.
Alla flaggorna går dock inte direkt från ALU:n. \mono{P}-flaggan använder inte
alltid den beräknade \mono{P}-flaggan som kommer från ALU:n. Vissa
instruktioner sätter den till värden som ALU:n inte har någon kännedom om.
Dessa är det nuvranande värdet av \mono{IFF} och om adressindatan till
registerfilen är noll. \mono{IFF}-värdet används endast för instruktionerna
\mono{ld a,i} och \mono{ld a,r}. Komparatorn används endast av
blockinstruktionerna för att indikera om \mono{BC}-1 är noll. De resterande
instruktionerna använder \mono{P}-flaggan som ALU:n har räknat ut.

\paragraph{ALU}
ALU:n tar emot två 8-bitars operander, de nuvarande flaggorna och en
instruktion --- samt en bit-select som används av endast vissa instruktioner,
de så kallade bitinstruktionerna. Med dessa indata avgörs därefter ett 8-bitars
resultat och en ny uppsättning av flaggor under samma klockintervall. ALU:n är
inte klockad och består endast av ett kombinatoriskt nät. Hela ALU:ns
uppbyggnad visas i blockschema \ref{diag:alu}. Gråskalan på bussarna indikerar
vilken bitbredd de har; 7, 8 eller 9 bitar. Inom ALU:n refereras operanderna
till \mono{op1} och \mono{op2}. \mono{op1} kommer från \mono{ACT} och
\mono{op2} kommer från \mono{TMP}. De olika flaggornas fulla namn och dess
användning beskrivs i sektion \ref{sec:theory:flags} på sidan
\pageref{sec:theory:flags}.

ALU:n räknar ut flera delresultat och resultat men använder sig av
instruktionen för att muxa vilka av dessa resultat som ska användas. En av
muxarna använder dessutom sig av den nuvarande \mono{C}-flaggan men de
resterande muxarna använder endast instruktionen som adress. Figur
\ref{fig:alu:sbc} visar exempelvis vilka delresultat som väljs ut när
instruktionen \mono{sbc} körs. \mono{sbc}-instruktionen utför beräkningen
\mono{op1}-\mono{op2}-\mono{C} och skickar ut det som resultat. I exemplet är
\mono{C}-flaggan satt sen innan. För aritmetiska instruktioner använder sig
ALU:n av en enda adderare men med olika möjliga operander. I exemplet kommer
\mono{op1} väljas och subtraheras med ett eftersom \mono{C} är satt. Den första
operanden till adderaren blir då \mono{op1}-1. För den andra operanden väljs
\mono{op2} och dess negation eftersom \mono{sbc} utför en subtraktion. Den
andra operanden är då \mono{-op2} och adderaren kommer ge
\mono{op1}-1-\mono{op2} vilket är det slutgiltiga resultatet som därefter muxas
till utdatan för resultatet.

\begin{SCfigure}
    \centering
    \includegraphics[width=0.7\linewidth]{alusbc.eps}
    \caption{Resultat och delresultat som muxas då \mono{sbc}-instruktionen
    utförs och \mono{C}-flaggan är satt.}
    \label{fig:alu:sbc}
\end{SCfigure}

\clearpage
\end{document}
