\documentclass[main.tex]{subfiles}

\begin{document}
\newpage
\subsubsection{ALU-sektion}
ALU-sektionen hanterar matematiska beräkningar samt hantering av flaggor.
Registerna \mono{ACT} och \mono{TMP}, ALU:n själv, en mux till \mono{P}-flaggan
samt en komparator för addressindatan tillhör ALU-sektionen. Hur de är
sammankopplade visas i mitten av blockschema \ref{diag:z80}.

\paragraph{\mono{ACT} och \mono{TMP}}
Det primära syftet av registerna \mono{ACT} och \mono{TMP} att lagra
operanderna under ALU:ns operation så att resultatet direkt kan läggas på
databussen. De två operanderna till ALU:n måste alltid laddas till \mono{ACT}
och \mono{TMP}. \mono{ACT} kallas för den temporära ackumulatorn och kan endast
ladda värden från \mono{A} i registerfilen. Vissa instruktioner använder endast
en operand, i så fall laddas den till \mono{TMP} och \mono{ACT}:s värde har
ingen påverkan på resultatet. \mono{TMP} kan ladda ett godtyckligt värde
eftersom den är ansluten till databussen. \mono{TMP} kan även skriva till
databussen och kan användas för temporär lagring, därav namnets ursprung.

\paragraph{Flaggor}
Flaggorna lagras i registerfilen men dess beräkning sker i ALU-sektionen. Vid
en ALU-instruktion får ALU:n alltid in de tidigare flaggorna direkt från
\mono{F} i registerfilen. Utifrån de tidigare flaggorna och resultatet beräknas
de nya flaggorna inom ALU:n. Under klockintervallet som ALU:n används läggs de
nya flaggorna på indatan till \mono{F} och lässignalen för \mono{F} går aktiv.
Alla flaggorna går dock inte direkt från ALU:n. \mono{P}-flaggan använder inte
alltid den beräknade \mono{P}-flaggan som kommer från ALU:n. Vissa
instruktioner sätter den till värden som ALU:n inte har någon kännedom om.
Dessa är det nuvranande värdet av \mono{IFF} och om adressindatan till
registerfilen är noll. \mono{IFF}-värdet används endast för instruktionerna
\mono{ld a,i} och \mono{ld a,r}. Komparatorn används endast av
blockinstruktionerna för att indikera om \mono{BC}-1 är noll. De resterande
instruktionerna använder \mono{P}-flaggan som ALU:n har räknat ut.

\paragraph{ALU}
ALU:n tar emot två 8-bitars operander, de nuvarande flaggorna och en
instruktion --- samt en bit-select som används av endast vissa instruktioner,
de så kallade bitinstruktionerna. Med dessa indata avgörs därefter ett 8-bitars
resultat och en ny uppsättning av flaggor under samma klockintervall. ALU:n är
inte klockad och består endast av ett kombinatoriskt nät. Hela ALU:ns
uppbyggnad visas i blockschema \ref{diag:alu}. Gråskalan på bussarna indikerar
vilken bitbredd de har; 7, 8 eller 9 bitar. Inom ALU:n refereras operanderna
till \mono{op1} och \mono{op2}. \mono{op1} kommer från \mono{ACT} och
\mono{op2} kommer från \mono{TMP}. De olika flaggornas fulla namn och dess
användning beskrivs i sektion \ref{sec:theory:flags} på sidan
\pageref{sec:theory:flags}.

ALU:n räknar ut flera delresultat och resultat men använder sig av
instruktionen för att muxa vilka av dessa resultat som ska användas. En av
muxarna använder dessutom sig av den nuvarande \mono{C}-flaggan men de
resterande muxarna använder endast instruktionen som adress.

\begin{SCfigure}
    \centering
    \includegraphics[width=0.7\linewidth]{alusbc.eps}
    \caption{Resultat och delresultat som muxas då \mono{sbc}-instruktionen
    utförs och \mono{C}-flaggan är satt.}
    \label{fig:alu:sbc}
\end{SCfigure}

Den översta delen i blockschemat visar beräkningen för aritmetiska
instruktioner. Aritmetiska instruktioner inkluderar bland annat \mono{add},
\mono{sub}, \mono{inc}, \mono{sbc} (subtract with carry), \mono{cp} (compare)
och \mono{daa} (decimal adjust after addition), Figur \ref{fig:alu:sbc} visar
exempelvis vilka delresultat som väljs ut när instruktionen \mono{sbc} körs.
\mono{sbc}-instruktionen utför beräkningen \mono{op1}-\mono{op2}-\mono{C} och
skickar ut det som resultat. I exemplet är \mono{C}-flaggan satt till 1 vid
instruktionen start. För aritmetiska instruktioner använder sig ALU:n av en
enda adderare men med olika möjliga operander. I exemplet kommer \mono{op1}
väljas och subtraheras med ett eftersom \mono{C} är satt. Den första operanden
till adderaren blir då \mono{op1}-1. För den andra operanden väljs \mono{op2}
och dess negation eftersom \mono{sbc} utför en subtraktion.  Den andra
operanden är då \mono{-op2} och adderaren kommer ge \mono{op1}-1-\mono{op2}
vilket är det slutgiltiga resultatet som därefter muxas till utdatan för
resultatet.

En annan grupp av instruktioner är bitinstruktionerna. De hanteras under
aritmetiken och skapar \mono{bit\_res} signalen. Denna grupp inkluderar
\mono{res}, \mono{set} och \mono{bit}-instruktioner. Dessa instruktioner
nollställer en bit, sätter en bit, respektive testar en bit. Dessa
instruktioner använder sig av \mono{bit\_select} som kommer direkt från
instruktionen i \mono{IR}. Ett exempel på en instruktion är \mono{set 6,e} som
sätter bit 6 i \mono{E} till ett.
\begin{align*}
    &\mono{F3: set 6,e}& \quad &
    {\underbrace{\mono{1 1}}_\mono{set}}\mono{ }
    {\underbrace{\mono{1 1 0}}_\mono{6}}\mono{ }
    {\underbrace{\mono{0 1 1}}_\mono{e}}
\end{align*}
För bitinstruktionerna skapas först en mask med hjälp av \mono{bit\_select}. I
fallet då \mono{bit\_select} är 6 skapas masken \mono{0100 0000}. Bit 6 är ett
och de resterande bitarna är noll. För att sätta bit 6 i ett godtyckligt tal
\mono{OR}:as tal med masken. För att nollställa bit 6 \mono{AND:as} talet med
inversen av masken. För att testa bit 6 \mono{AND}:as talet med masken och
resultatet jämförs med noll.

Nästa grupp av instruktioner är rotations- och skiftinstruktionerna. Dessa är
bland annat \mono{rr} (rotate right), \mono{sla} (shift left arithmetic) och
\mono{rlc} (rotate left with carry). Dessa skiftar talet antingen ett steg åt
höger eller vänster. Den nya biten som tillkommer, \mono{edge}, skiljer får
olika värden beroende på instruktionen. Vid rotation kan biten som kom ut från
andra sidan användas, vid skiftning kan till exempel noll tillkomma. De olika
instruktioner skiljer också med vilket värde som hamnar i den nya
carry-flaggan. Alla dessa instruktioners funktion beskrivs i
användarmanualen\cite{z80um} i kapitlet som börjar på sidan 204.

Den sista delen av ALU:n används för endast två instruktioner; \mono{rld} och
\mono{rrd}. Dessa utför en rotation inom ett 16-bitars tal. Talet består av
\mono{A} som de högre bitarna och värdet som \mono{HL} pekar på i minnet som de
lägre bitarna. Instruktionerna roterar fyra bitar i tagen höger för \mono{rrd}
och vänster för \mono{rld}. Eftersom resultatet är ett 16-bitars tal kan ALU:n
inte räkna ut det under ett klockintervall. Istället räknas talet ut i två
steg; först de lägre åtta bitarna och därefter de högre åtta bitarna.

Det finns en annan grupp av instruktioner som också använder ALU:n i två steg.
Dessa är \mono{add}, \mono{adc} och \mono{sbc} för 16-bitars register, till
exempel \mono{add ix, bc}. Dessa tar två 16-bitars tal och ger ett nytt
16-bitars tal. I fallet för \mono{add} sker här först en vanlig
\mono{add}-instruktion för de lägre åtta bitarna av varje register. Under det
andra steget körs istället en \mono{adc}-instruktion för de högre åtta bitarna.
\mono{adc} använder carry:n från det första steget, på så sätt försvinner inget
värde om det första steget ger ett tal större än 256.

\clearpage
\end{document}
