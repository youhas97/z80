\documentclass[main.tex]{subfiles}

\begin{document}
\subsubsection{Kontrollsektion}
Kontrollsektionen består av tre komponenter; instruktionsregistret,
tillståndsmaskinen och instruktionsdekodaren. Tillståndsmaskinen lagrar och
ändrar hela processorns tillstånd, förutom alla registervärden. Utifrån
processorns tillstånd och instruktionregistrets värde aktiverar
instruktionsdekodaren, som endast är ett stort kombinatoriskt nät, olika
kontrollsignaler.

\paragraph{Instruktionsregistret \mono{IR}}
\mono{IR} är det register som instruktioner direkt laddas till när de läses
från minnet. Här lagras de fram till att nästa instruktion laddas in. Om en
instruktion har ett prefix lagras först prefixet i \mono{IR} och skrivs kort
därefter över med ett nytt prefix eller en instruktion.

\paragraph{Tillstånd}
Tillståndsmaskinen har fem olika tillståndsvariabler; läget \mono{mode},
avbrottsläget \mono{im}, instruktionsprefixet \mono{prefix}, maskincykeln
\mono{M} och T-tillståndet \mono{T}. Tillståndsmaskinen lagrar dessa och sänder
dem till instruktionskodaren. I blockschema \ref{diag:z80} är alla dessa
signaler inkluderade i signalen \mono{cpu\_state}.

\begin{SCfigure}
    \centering
    \begin{tikztimingtable}
    \mono{clk}     & [c]62{c} \\
    \mono{M}       & 4D{1} 4D{2} 3D{3} 4D{1} 3D{2} 13D{1} \\
    \mono{T}       & D{1}D{2}D{3}D{4} D{1}D{2}D{3}D{4} D{1}D{2}D{3} % add rp
                     D{1}D{2}D{3}D{4} D{1}D{2}D{3} D{4} % im 1
                     D{1}D{2}D{3} D{4} % halt fetch
                     D{1}D{2}D{3}D{4} % halt
                     D{1}D{2}D{3}D{4} % halt
                     \\
    \mono{mode}    & 23D{\mono{exec}} 8D{\mono{halt}} \\
    \mono{im}      & 19D{0} 12D{1} \\
    \mono{prefix}  & 18D{\mono{main}} 4D{\mono{ed}} 9D{\mono{main}} \\
    \mono{ir}      & 3D{}  11D{\mono{39}} 4D{\mono{ED}} 4D{\mono{56}}
                     9D{\mono{76}} \\
\end{tikztimingtable}


    \caption{Tillstånd samt värdet av \mono{IR} under exekveringen av
    programmet \mono{add hl,sp; im 1; halt}}
    \label{fig:tim-states}
\end{SCfigure}

\mono{T} är ett heltal mellan 1 och 6 som avgör vilken klockpuls av en
maskincykel processorn är i. En maskincykel kan bestå av 3 till 6 klockpulser.
Med andra ord ökas \mono{T} med 1 varje klockpuls förutom när processorn byter
maskincykel, då sätts \mono{T} till 1 igen.

\mono{M} är ett heltal mellan 1 och 5 men som avgör vilken del av en
instruktion processorn består i. Vid instruktioner som består av flera ord
kommer maskincykeln gå tillbaka till 1 vid varje hämtning. Detta är för att
göra instruktionsdekodaren enklare. Instruktionshämtning sker därmed endast på
maskincykel 1. Det innebär dessutom att liknande instruktioner med olika prefix
kan återanvändas eftersom allas exekveringsfas börjar på maskincykel 1. Till
exempel instruktionerna \mono{add hl,sp} (39) och \mono{add ix,sp} (DD39) är
identiska förutom att de har förskjutna maskincyklar och destinationsregistret
har olika addresser. Deras maskincyklar visas i figur \ref{fig:mcorr}.

\begin{figure}[b]
    \center
    \begin{tabular}{|r|p{2cm}|p{2cm}|p{2cm}|p{2cm}|}
        \hline
        maskincykel & 1 & 2 & 3 & 4 \\ \hline
        \multicolumn{5}{l}{\mono{add hl,sp}} \\ \hline
        prefix      & \multicolumn{4}{l|}{\mono{main}} \\ \hline
        fas         & hämta 39 & addera låga & addera höga & färdig \\ \hline
        \multicolumn{5}{l}{\mono{add ix,sp}} \\ \hline
        prefix      & \mono{main} & \multicolumn{3}{l|}{\mono{dd}} \\ \hline
        fas         & hämta DD & hämta 39 & addera låga & addera höga \\ \hline
    \end{tabular}
    \caption{tidsdiagram för instruktionerna \mono{add hl,sp} och
            \mono{add ix,sp}} utan korrigering av maskincyklar.
    \label{fig:mcorr}
\end{figure}

Genom att sätta maskincykeln till 1 vid andra hämtningen hamnar maskincyklarna
i fas och instruktionsdekodaren kan ge identiska kontrollsignaler för båda
instruktioner utöver addressen till destinationsregistret.

Prefix-tillståndet tillsammans med \mono{IR} används för att avgöra vilken
instruktion som ska exekveras. Det finns åtta olika prefix; \mono{main} för
huvudinstruktioner, \mono{ed} för utökade instruktioner, \mono{cb} för
bitinstruktioner, \mono{dd}/\mono{fd} för IX/IY-instruktioner,
\mono{ddcb}/\mono{fdcb} för IX/IY-bitinstruktioner samt \mono{int} som sätts
vid avbrott. Vid processorns start och efter varje instruktion sätts prefixet
till \mono{main}. Om \mono{IR} därefter laddas med en prefix-byte kommer
prefix-tillståndet att övergå till motsvarande prefix och maskincykeln sätts
till 1 så att hämtningsfasen börjar om på nytt. Därefter kan ett nytt prefix
eller en instruktion laddas. Om en instruktion laddas påbörjas dess exekvering
direkt. I ovan exempel är prefix-byten \mono{DD} och instruktions-byten
\mono{39}.

Det finns dessutom ett prefix som inte går att komma åt med ett prefix i
instruktionen. Det är \mono{int}-prefixet. Detta sätts alltid efter en
instruktion har avslutats om ett avbrott har skett. Om prefixet är \mono{int}
kommer instruktionsdekodaren inte exekvera någon instruktion utefter vad
\mono{IR} innehåller. Istället kommer den då alltid att köra en avbrottssekvens
utefter vad \mono{im}-tillståndet är. \mono{im}-tillståndet avgör det
nuvarande avbrottsläget och kan väljas av programmeraren med \mono{im 0},
\mono{im 1} och \mono{im 2}-instruktionerna.

\begin{figure}
    \center
    \tikzset{
    state/.style={
        circle,
        thick,
        draw=black,
        fill=gray!20,
        minimum size=10mm,
    },
}

\begin{tikzpicture}[node distance=1cm and 1.5cm, font=\scriptsize]
    \node[state] (main) {\mono{main}};
    \node[state] (dd) [below right=of main] {\mono{dd}};
    \node[state] (ddcb) [above right=of dd] {\mono{ddcb}};
    \node[state] (fd) [below left=of main] {\mono{fd}};
    \node[state] (fdcb) [above left=of fd] {\mono{fdcb}};
    \node[state] (ed) [above left=of main] {\mono{ed}};
    \node[state] (cb) [above right=of main] {\mono{cb}};
    \node[state] (int) [above=of main] {\mono{int}};

    \path[->, thick] (main) edge[loop below] node{} ();

    \path[->, thick] (main) edge node[above right]{\mono{IR=DD}} (dd);
    \path[->, thick] (dd) edge[bend left] node{} (main);
    \path[->, thick] (dd) edge node[below right]{\mono{IR=CB}} (ddcb);
    \path[->, thick] (ddcb) edge node{} (main);

    \path[->, thick] (main) edge node[above left]{\mono{IR=FD}} (fd);
    \path[->, thick] (fd) edge[bend right] node{} (main);
    \path[->, thick] (fd) edge node[below left]{\mono{IR=CB}} (fdcb);
    \path[->, thick] (fdcb) edge node{} (main);

    \path[->, thick] (main) edge[bend left] node[left]{\mono{IR=ED}} (ed);
    \path[->, thick] (ed) edge node{} (main);

    \path[->, thick] (main) edge[bend right] node[right]{\mono{IR=CB}} (cb);
    \path[->, thick] (cb) edge node{} (main);

    \path[->, thick] (main) edge node[above left]{\mono{mode=int}} (int);
    \path[->, thick] (int) edge[bend left] node{} (main);
\end{tikzpicture}

    \caption{Tillståndsdiagram för prefixtillståndet.}
    \label{fig:prefix}
\end{figure}

Prefixet uppdateras med lässignalen till \mono{IR}. Det nästa prefixet beror på
det föregående värdet i \mono{IR} som kan ses i figur \ref{fig:tim-states} vid
övergången från \mono{main} till \mono{ed}.

Det sista tillståndet i processorns tillståndsmaskin är \mono{mode} som kan
anta de tre olika tillstånden \mono{exec}, \mono{halt} och \mono{interrupt}.
Normalt sätt är processorn i \mono{exec}. Då kommer instruktionsdekodaren
exekvera instruktioner utefter \mono{IR} prefixet och även \mono{IM}. I
\mono{halt} däremot kommer processorn fortsätta exekvera instruktioner men den
hämtar inte nya instruktioner under maskincykel 1. Normalt sätt hamnar
processorn i \mono{halt} från \mono{HALT}-instruktionen. När \mono{HALT}
instruktionen avslutas övergår processorn till \mono{halt}-läge och
maskincykel 1 men eftersom den inte hämtar en ny instruktion kommer den
exekvera \mono{HALT} om och om igen tills processorn får ett avbrott eller
återställs. Det sista läget är \mono{interrupt} som väljs när ett avbrott
sker. Anledningen till att det finns både ett \mono{interrupt}-läge och ett
\mono{int}-prefix är för att även om ett avbrott har skett kan en överlappande
instruktion behöva exekvera färdigt innan avbrottsekvensen körs.
\mono{interrupt}-läget förhindrar behöver förhindra att en ny instruktion som
inte ännu ska exekveras hämtas och att programräknaren inte ökas innan den ska
sparas till stacken.

\paragraph{Kontrollsignaler}
Kontrollsignalerna är uppdelade i tre grupper; extern kontrollbus,
tillståndskontroll samt interna kontrollsignaler.

% TODO diagram för timing under read/write (kanske inte behövs)
% kontrollbuss
Den externa kontrollbussen har sex utsignaler som instruktionsdekodaren
aktiverar; \mono{M1}, \mono{MREQ}, \mono{IORQ}, \mono{RD} och \mono{WR}.
\mono{RD} och \mono{WR} används för att signalera läsning eller skrivning till
och från antingen minnet eller IO-portar. Processorn signalerar vilket med
hjälp av \mono{MREQ} och \mono{IORQ}. \mono{M1} används för att signalera om
CPU:n är i maskincykel 1.

% tillståndskontroll
Kontrollsignalerna för tillståndsmaskinen avgör processorns nästa tillstånd.
Signalerna är \mono{set\_m1}, \mono{cycle\_end}, \mono{instr\_end},
\mono{mode\_next} och \mono{im\_next}. I blockschema \ref{diag:z80} är dessa
samlade under signalen \mono{ctrl}.

\begin{SCfigure}
    \centering
    \begin{tikztimingtable}
    \mono{M}       & 4D{1} 4D{2} 3D{3} 4D{1} 3D{2} 13D{1} \\
    \mono{T}       & D{1}D{2}D{3}D{4} D{1}D{2}D{3}D{4} D{1}D{2}D{3} % add rp
                     D{1}D{2}D{3}D{4} D{1}D{2}D{3} D{4} % im 1
                     D{1}D{2}D{3} D{4} % halt fetch
                     D{1}D{2}D{3}D{4} % halt
                     D{1}D{2}D{3}D{4} % halt
                     \\
    \mono{cycle\_end}   & 3L1H 3L1H 2L1H 3L1H 3L1H 3L1H 3L1H 3L1H \\
    \mono{set\_m1}      & 4L 4L 3L 4L 2L1H 13L \\
    \mono{instr\_end}   & 4L 4L 2L1H 4L 3L1H 3L1H 3L1H 3L1H \\
    \mono{mode\_next}   & 22D{\mono{exec}} 9D{\mono{halt}} \\
    \mono{mode}    & 23D{\mono{exec}} 8D{\mono{halt}} \\
    \mono{im\_next}     & 18D{0} 13D{1} \\
    \mono{im}      & 19D{0} 12D{1} \\
    \mono{prefix}  & 18D{\mono{main}} 4D{\mono{ed}} 9D{\mono{main}} \\
    \mono{ir}      & 3D{}  11D{\mono{39}} 4D{\mono{ED}} 4D{\mono{56}}
                     9D{\mono{76}} \\
\end{tikztimingtable}

    \caption{Programmet från figur \ref{fig:tim-states} med kontrollsignalerna
    för tillstånden synliga.}
    \label{fig:tim-statectrl}
\end{SCfigure}

\mono{set\_m1} används för syftet beskrivet ovan; att synkronisera liknande
instruktioner för att kunna återanvända hårdvara i instruktionsdekodaren till
flera instruktioner. Den används endast efter hämtning av prefix så att varje
instruktion alltid börjar exekvera under \mono{M1T4}

\mono{cycle\_end} används för att signalera att den nuvarande cykeln är
avslutad. I slutet av varje maskincykel går \mono{cycle\_end} aktiv så att
tillståndsmaskinen återställer \mono{T}-tillståndet till \mono{T1} och ökar
\mono{M} med ett. På liknande sätt går \mono{instr\_end} alltid aktiv i slutet
av en instruktion. När den går aktiv återställs maskincykeln till \mono{M1},
\mono{mode} antar värdet av \mono{mode\_next} och \mono{im} antar värdet av
\mono{im\_next}. Notera dock att instruktionen fortsätter att exekvera under
nästa maskincykel efter att \mono{instr\_end} har gått aktiv. Först när
\mono{IR} har laddats med nästa instruktion (T4) är det nästa instruktion som
exekveras. \mono{mode\_next} och \mono{im\_next} vanligtvis antar det nuvarande
värdet av \mono{mode} respektive \mono{im} och antar endast ett annat värde då
\mono{instr\_end} går aktiv om ett byte ska ske.

% kontrollsignaler TODO
Processorn har även en grupp kontrollsignaler för alla andra komponententer
inuti processorn som alla tillhör ett långt kontrollord. För att ge en
överblick inför de resterande två sektionerna följer en kort beskrivning av
varje signal i kontrollordet.

\begin{labeling}{indentzzZzZz}
\item[\mono{dbus\_src}]
    Komponentens utdata som ska muxas till databussen. Kan komma från
    \mono{dbufi}, registerfilen, \mono{TMP} eller ALU:n samt en sträng av
    nollor.
\item[\mono{abus\_src}]
    Källan till adressbussen; registerfilen, adressadderaren eller
    \mono{rst}-adressen.
\item[\mono{rf\_daddr}]
    Adressen till ett 8-bitars register i registerfilen vars värde läggs som
    utdata mot databussen.
\item[\mono{rf\_addr}]
    Adressen till ett 16-bitars register i registerfilen vars värde läggs som
    utdata mot adressbussen och adressadderaren.
\item[\mono{rf\_rdd}]
    Läs från databussen och skriv värdet till registret som \mono{rf\_daddr}
    pekar på.
\item[\mono{rf\_rda}]
    Läs från adressbussen och skriv värdet till registret som \mono{rf\_aaddr}
    pekar på.
\item[\mono{rf\_swp}]
    Utför ett byte i registerfilen. Signalen kan anta något av \mono{none},
    \mono{af}, \mono{reg}, \mono{dehl} och \mono{afwz}. \mono{af} växlar
    värdena på \mono{AF} och \mono{AF'}.  \mono{reg} växlar värdena på
    \mono{BC} med \mono{BC'}, \mono{DE} med \mono{DE'} och \mono{HL} med
    \mono{HL'}. \mono{dehl} växlar värdena av \mono{DE} och \mono{HL}.
    \mono{afwz} växlar värdena mellan \mono{AF} och \mono{WZ}.
\item[\mono{rf\_ldpc}]
    Ladda värdet av registret som \mono{rf\_aaddr} pekar på till \mono{PC} i
    registerfilen.
\item[\mono{f\_rd}]
    Läs flaggor till \mono{F} som ligger i registerfilen.
\item[\mono{pv\_src}]
    Var värdet av \mono{P}-flaggan ska komma ifrån.
\item[\mono{ir\_rd}]
    Läs från databussen till instruktionsregistret.
\item[\mono{addr\_op}]
    Operation som ska utföras på adressbussens värde innan det går till
    registerfilen. Kan vara \mono{none}, \mono{inc} eller \mono{dec}.
\item[\mono{rst\_addr}]
    Bitar 3-5 av adressen vid en \mono{rst}-instruktion.
\item[\mono{iff\_next}]
    Nästa värde av \mono{iff}, D-vippan för avbrott.
\item[\mono{alu\_op}]
    Instruktionen som ska utföras av ALU:n.
\item[\mono{act\_rd}]
    Läs värdet av \mono{A} i registerfilen till \mono{ACT}.
\item[\mono{tmp\_rd}]
    Läs värdet från databussen till \mono{TMP}.
\item[\mono{data\_rdi}]
    Läs från den externa databussen till buffern \mono{dbufi}.
\item[\mono{data\_rdo}]
    Läs från databussen till buffern \mono{dbufo}.
\item[\mono{data\_wro}]
    Skriv värdet från \mono{dbufo} till den externa databussen.
\item[\mono{addr\_rd}]
    Läs från adressbussen till buffern \mono{abuf}.
\end{labeling}

\end{document}
