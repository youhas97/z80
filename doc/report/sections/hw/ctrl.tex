\documentclass[main.tex]{subfiles}

\begin{document}
\subsubsection{Kontrollsektion}
Kontrollsektionen består av tre komponenter; instruktionsregistret,
tillståndsmaskinen och instruktionsdekodaren. Tillståndsmaskinen lagrar och
ändrar hela processorns tillstånd, förutom alla registervärden. Utifrån
processorns tillstånd och instruktionregistrets värde aktiverar
instruktionsdekodaren, som endast är ett stort kombinatoriskt nät, olika
kontrollsignaler.

\paragraph{Instruktionsregistret \mono{IR}}
\mono{IR} är det register som instruktioner direkt laddas till när de läses
från minnet. Här lagras de tills nästa instruktion laddas in. Om en instruktion
har ett prefix lagras först prefixet i \mono{IR} och skrivs kort därefter över
med ett nytt prefix eller en instruktion.

\paragraph{Tillstånd}
Tillståndsmaskinen har fem olika tillståndsvariabler; läget \mono{mode},
avbrottsläget \mono{IM}, instruktionsprefixet \mono{prefix}, maskincykeln
\mono{M} och T-tillståndet \mono{T}.

\mono{T} är ett heltal mellan 1 och 6 som avgör vilken klockpuls av en
maskincykel processorn är i. En maskincykel kan bestå av 3 till 6 klockpulser.
Med andra ord ökas \mono{T} med 1 varje klockpuls förutom när processorn byter
maskincykel, då sätts \mono{T} till 1 igen.

\mono{M} är också ett heltal mellan 1 och 5 men som avgör vilken del av en
instruktion processorn består i. Vid instruktioner som består av flera ord
kommer maskincykeln gå tillbaka till 1 vid varje hämtning. Detta är för att
göra instruktionsdekodaren enklare. Instruktionshämtning sker därmed endast på
maskincykel 1. Det innebär dessutom att liknande instruktioner med olika prefix
kan återanvändas eftersom allas exekveringsfas börjar på maskincykel 1. Till
exempel instruktionerna \mono{ADD HL, SP} och \mono{ADD IX, SP}.
Instruktionernas objektkoder är \mono{39} respektive \mono{DD39}. Båda instruktioner är
identiska förutom att de har förskjutna maskincyklar och destinationsregistret
har olika addresser.

\begin{figure}[H]
    \center
    \begin{tabular}{|r|p{2cm}|p{2cm}|p{2cm}|p{2cm}|}
        \hline
        maskincykel & 1 & 2 & 3 & 4 \\ \hline
        \multicolumn{5}{l}{\mono{ADD HL, SP}} \\ \hline
        prefix      & \multicolumn{4}{l|}{\mono{main}} \\ \hline
        fas         & hämta 39 & addera låga & addera höga & färdig \\ \hline
        \multicolumn{5}{l}{\mono{ADD IX, SP}} \\ \hline
        prefix      & \mono{main} & \multicolumn{3}{l|}{\mono{DD}} \\ \hline
        fas         & hämta DD & hämta 39 & addera låga & addera höga \\ \hline
    \end{tabular}
\end{figure}

Genom att sätta maskincykeln till 1 vid andra hämtningen hamnar maskincyklarna
i fas och instruktionsdekodaren kan ge identiska kontrollsignaler för båda
instruktonerna utöver addressen till destinationsregistret.

Som ovan exempel visar används prefix-tillståndet tillsammans med \mono{IR} för att
avgöra vilken instruktion som ska exekveras. Det finns åtta olika prefix;
\mono{main} för huvudinstruktioner, \mono{ED} för utökade instruktioner,
\mono{CB} för bitinstruktioner, \mono{DD}/\mono{FD} för IX/IY-instruktioner,
\mono{DDCB}/\mono{FDCB} för IX/IY-bitinstruktioner samt \mono{int} som sätts
vid avbrott. Vid processorns start och efter varje instruktion sätts prefixet
till \mono{main}. Om \mono{IR} därefter laddas med en prefix-byte kommer
prefix-tillståndet att övergå till motsvarande prefix och maskincykeln sätts
till 1 så att hämtningsfasen börjar om på nytt. Därefter kan ett nytt prefix
eller en instruktion laddas. Om en instruktion laddas påbörjas dess exekvering
direkt. I ovan exempel är prefix-byten \mono{DD} och instruktions-byten
\mono{39}.

Det finns dessutom ett prefix som inte går att komma åt med ett prefix i
instruktionen. Det är \mono{int}-prefixet. Detta sätts alltid efter en
instruktion har avslutats om ett avbrott har skett. Om prefixet är \mono{int}
kommer instruktionsdekodaren inte exekvera någon instruktion utefter vad
\mono{IR} innehåller. Istället kommer den då alltid att köra en avbrottssekvens
utefter vad \mono{IM}-tillståndet är. \mono{IM}-tillståndet avgör det
nuvarande avbrottsläget och kan väljas av programmeraren med \mono{IM 0},
\mono{IM 1} och \mono{IM 2}-instruktionerna.

Det sista tillståndet i processorns tillståndsmaskin är \mono{mode} som kan
anta de tre olika tillstånden \mono{exec}, \mono{halt} och \mono{interrupt}.
Normalt sätt är processorn i \mono{exec}. Då kommer instruktionsdekodaren
exekvera instruktioner utefter \mono{IR}, prefixet och även \mono{IM}. I
\mono{halt} däremot kommer processorn fortsätta exekvera instruktioner men den
hämtar inte nya instruktioner under maskincykel 1. Normalt sätt hamnar
processorn i \mono{halt} från \mono{HALT}-instruktionen. När \mono{HALT}
instruktionen  avslutas övergår processorn till \mono{halt}-läge och
maskincykel 1 men eftersom den inte hämtar en ny instruktion kommer den
exekvera \mono{HALT} igen tills processorn får ett avbrott eller återställs.
Det sista läget är \mono{interrupt} som väljs när ett avbrott sker. Anledningen
till att det finns både ett \mono{interrupt}-läge och ett \mono{int}-prefix är
för att även om ett avbrott har skett kan en överlappande instruktion behöva
exekvera färdigt innan avbrottsekvensen körs. \mono{interrupt}-läget förhindrar
dessutom att en ny instruktion som inte ännu ska exekveras hämtas och att
programräknaren inte ökas innan den ska sparas till stacken.

\paragraph{Kontrollsignaler}
Kontrollsignalerna är uppdelade i tre grupper; extern kontrollbus,
tillståndskontroll samt interna kontrollsignaler.

Den externa kontrollbussen har sex utsignaler som instruktionsdekodaren
aktiverar; \mono{M1}, \mono{MREQ}, \mono{IORQ}, \mono{RD} och \mono{WR}.
\mono{RD} och \mono{WR} används för att signalera läsning eller skrivning till
och från antingen minnet eller IO-portar. Processorn signalerar vilket med
hjälp av \mono{MREQ} och \mono{IORQ}. \mono{M1} används för att signalera om
CPU:n är i maskincykel 1. Om processorn läser från minnet under maskincykel 1
vet externa enheter att processorn hämtar en instruktion.

Kontrollsignalerna för tillståndsmaskinen avgör processorns nästa tillstånd.

\paragraph{Faser}

\end{document}
