\documentclass[main.tex]{subfiles}

\newcommand{\diagw}{0.7\textwidth}

\begin{document}
\subsubsection{Instruktionsförlopp}
För att sammanfatta processorns implementation följer ett exempel av hela
händelseförloppet för en instruktion. Instruktionen är \mono{add a,d} och den adderar
\mono{A} med \mono{D} och lagrar resultatet i \mono{A}.

\paragraph{\mono{add a,d}: \mono{M1T1}}
\begin{figure}[H]
    \center
    \includegraphics[width=\diagw]{m1t1.eps}
\end{figure}
Processorn börjar på maskincykel 1 och T-state 1. Det första som måste ske är
att instruktionen hämtas från minnet. Adressen till \mono{PC} väljs i
registerfilen. Registerfilens utdata muxas till adressbussen.  När \mono{PC}
väl ligger på databussen kan det läsas till adressbuffern.

\paragraph{\mono{add a,d}: \mono{M1T2}}
\begin{figure}[H]
    \center
    \includegraphics[width=\diagw]{m1t2.eps}
\end{figure}
Inför T2 har adressbuffern laddat adressen och kan nu skicka den till minnet. I
och med det skickas en lässignal så att instruktionen i minnet ligger på den
externa databussen i slutet av klockintervallet. Därifrån läses den till
processorns indatabuffer \mono{dbufi}. Under samma klockintervall behålls
\mono{PC} på adressbussen för att inkrementeras så att PC nu pekar på nästa
instruktion.

\paragraph{\mono{add a,d}: \mono{M1T3}}
\begin{figure}[H]
    \center
    \includegraphics[width=\diagw]{m1t3.eps}
\end{figure}
Nu återstår bara att läsa instruktionen från databuffern till \mono{IR} där
instruktionsdekodaren kan avläsa den.

\paragraph{\mono{add a,d}: \mono{M1T4}}
\begin{figure}[H]
    \center
    \includegraphics[width=\diagw]{m1t4.eps}
\end{figure}
Först nu ligger objektkoden för \mono{add a,d} i \mono{IR} och processorn kan
börja exekvera instruktionen. Det första som sker är att \mono{TMP} laddas med
\mono{D} och \mono{ACT} med \mono{A}. \mono{A} har en egen buss och kan därmed
läsas parallellt med ett annat värde som går via databussen.
Instruktionsdekodaren signalerar nu att masknincykeln är färdig och att
hämtningen av nästa instruktion kan påbörja. \mono{add a,d}-instruktion är dock
inte färdig riktigt ännu.

\paragraph{\mono{add a,d}: \mono{M2T1}}
Precis som under \mono{M1T1} läggs \mono{PC} på adressbussen och laddas till
adressbuffern. \mono{PC} pekar dock nu på instruktionen som ligger på platsen
efter \mono{add a,d}.

\paragraph{\mono{add a,d}: \mono{M2T2}}
\begin{figure}[H]
    \center
    \includegraphics[width=\diagw]{m2t2.eps}
\end{figure}
Under tiden som instruktionshämtningen sker nere på adressbussen kan \mono{add
a,d}-instruktionen avslutas. Nu beräknar ALU:n \mono{A}+\mono{B} och lägger det
på databussen. Därifrån läses det av registerfilen till \mono{A}.
Registerfilen läser även in de nya flaggorna från en dedikerad buss. Under
tiden som ALU:n beräknar laddas den nya instruktionen till \mono{dbufi}. 

\paragraph{\mono{add a,d}: \mono{M2T3}}
Precis som under \mono{M1T3} läses den nya instruktionen från databuffern in
till \mono{IR}. \mono{add a,d} är härmed avslutad och under nästa
klockintervall kommer nästa instruktion att börja exekvera.

\end{document}
