\documentclass[main.tex]{subfiles}

\begin{document}
\subsubsection{Avbrott}
Konstruktionen använder sig av avbrott från tre källor; hårdvarutimer 1,
hårdvarutimer 2 och ON-knappen. TI-83p har även ett avbrott från länkporten men
denna är ej implementerad. TI-ASIC:en har en avbrottshanterare som tar emot
signaler från ON-knappen och ASIC:ens hårdvarutimers och därefter avgör om
\mono{int}-signalen ska gå aktiv utefter avbrottsmasken.

Signalen från ON-knappen kommer direkt från tangentbordskodaren. Om knappen är
nedtryckt är signalen låg, annars är den hög. Signalerna för timers:en
genereras i en egen komponent. Komponenten består av två räknare som alltid
räknar ner. När en räknare kommer till noll skickas en puls till
avbrottshanteraren och räknaren laddas om med sitt startvärde. Startvärdet
beror på vilken frekvens som är inställd och frekvensen på räknaren som är
\SI{33}{\mega\hertz}. Vid start ska timer 1 ha frekvensen \SI{118}{\hertz}, då
måste startvärdet vara $\frac{\SI{33}{\mega\hertz}}{\SI{118}{\hertz}} =
282486$. Då kommer räknaren skicka en puls 118 gånger per sekund. Frekvensen på
timers:en går att välja med genom port \mono{04}.

\begin{SCfigure}
    \centering
    \tikzset{
    state/.style={
        circle,
        thick,
        draw=black,
        fill=gray!20,
        minimum size=20mm,
    },
}

\begin{tikzpicture}[node distance=1cm and 3cm, font=\normalsize]
    \node[state] (off) {\mono{inactive}};
    \node[state] (on) [right=of off] {\mono{fire}};
    \node (off-left) [left=of off] {};

    \path[->, ultra thick] (off-left) edge (off);

    \path[->, ultra thick] (off) edge[loop below] node{} ();
    \path[->, ultra thick] (off) edge[bend left]
        node[above=12mm]{\mono{enable} = '1'} (on);
    \path[->, ultra thick] (off) edge[bend left]
        node[above=7mm]{and} (on);
    \path[->, ultra thick] (off) edge[bend left]
        node[above=2mm]{\mono{activate} = '1'} (on);

    \path[->, ultra thick] (on) edge[loop below] node{} ();
    \path[->, ultra thick] (on) edge[bend left]
        node[below]{\mono{enable} = '0'} (off);
\end{tikzpicture}

    \caption{Tillståndsdiagram för avbrottshantering av en enskild källa.}
    \label{fig:prefix}
\end{SCfigure}

Varje avbrottskälla hanteras individuellt i avbrottshanteraren. Det finns två
signaler som avgör om ett avbrott ska avfyras; \mono{enable} och
\mono{activate}. \mono{enable} är biten för avbrottskällan i avbrottsmasken i
port \mono{03}. \mono{activate} är en signal från själva avbrottskällan, till
exempel pulsen från en timer. Om \mono{activate} går hög måste \mono{enable}
vara hög för att avbrottet ska avfyras. När avbrottet väl ha avfyrats kommer
det vara aktivt till att biten i masken nollställs. Port \mono{04} visar vilka
avbrott som är aktiva. Det räcker med att ett avbrott är aktivt för att
\mono{int} ska gå hög. De tre källorna \mono{OR:as}. Programmeraren kan först
kolla vilket avbrott som är källan och därefter inaktivera just det avbrottet
innan avbrott för processorn sätts på igen med \mono{ei}-instruktionen.

\end{document}
