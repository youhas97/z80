\documentclass[main.tex]{subfiles}

\begin{document}
\subsubsection{Tangentbordskodare}
Tangentbordskodaren har i uppgift att översätta knapptrycken på
PS/2-tangentbordet till miniräknarens format som beskrivs i sektion
\ref{theory:kbd} på sidan \pageref{theory:kbd}. Vilken knapp på PS/2
tangentbordet som motsvarar vilken på miniräknaren visas i tabell
\ref{app:kbd}.

Tangentbordet skickar endast signaler vid nedtryckning och uppsläpp av
tangenter. Därför måste tangentbordskodaren lagra en matris av vippor som
håller koll på varje tangents nuvarande status. När en knapp trycks ner sätts
motsvarande vippa till noll och när en knappen släpps sätts vippan till ett
igen. Denna matris är direkt kopplad till tangentbordskontrollern i TI-ASIC:en
som AND:ar de grupper som programmeraren har aktiverat.
\end{document}
