\documentclass[main.tex]{subfiles}

\begin{document}
\subsubsection{Tangentbordskodare}
Tangentbordskodaren översätter kodningen på tangentbordets knapptryckningar
till en annan kodning som operativsystemet för TI-83 använder sig av. En
multiplexer används för att bestämma värdet på utdatan från tangentbordskodaren
utifrån ett femtiotal möjliga värden på indata. När en tangent på tangentbordet
trycks ner returneras en bit från tangentbordet. Den här biten används för att
bestämma vilken grupp av tangenter som använts på tangentbordet med hjälp av en
tabell \cite{kbdenc}. Efter att ha bestämt vilken grupp av tangenter som
använts är det lättare att se vilken tangent som tryckts när då det nu finns
färre tangenter att kolla. För att hålla kontruktionen så enkel som möjligt
angående upplägget så har nummerkontrollerna på tangentbordet använts som
siffror. Utöver det har knappar som varit lättast att implementera använts.

I tangentbordkontrollen har en matris skapats. Den här matrisen används för att
kunna hantera flera knapptryckningar på samma gång. Matrisen innehåller alla
tangenter i form av ettor, som sedan ändras till nollor då knappen trycks ner.
Genom att gå igenom matrisen kan man se vilka tangenter som tryckts ner och
flera aktioner kan på så sätt göras simultant. Endast de utmaskade gruppernas
tangenter behöver dock kollas i matrisen då man redan uteslutit vilka tangenter
som används/inte används. Matrisen hanteras av tangentbordskontrollern.
\end{document}
