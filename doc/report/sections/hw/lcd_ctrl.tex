\documentclass[main.tex]{subfiles}

\begin{document}
\subsubsection{LCD-kontroller}
Implementationen av LCD-kontrollern består av räknare och register för
\mono{X}, \mono{Y} och \mono{Z} samt ett block-RAM på 7680 bitar. Tre D-vippor
används även för att hålla koll på läget (\mono{wl}) och pekarens
rörelseinställningar; upp/ner (\mono{up}) för \mono{X}/\mono{Y}
(\mono{counter}. Det finns även ett register som lagrar bildminnets utdata för
läsning från processorn.

\mono{X} består av en upp/ner räknare på sex bitar. Den går upp till 63 och sen
går runt till noll igen vid uppräkning. Likaså går den från noll till 63 vid
nedräkning. Dess \mono{CE}-ingång är kopplad till läs/skriv-signalen från port
\mono{11} AND:ad med D-vippan som väljer mellan \mono{X} och \mono{Y}. På så
sätt räknar \mono{X} endast om \mono{X} är vald samt att porten skrivs eller
läses ifrån. \mono{U/D}-ingången som väljer riktning är direkt kopplad till
D-vippan som väljer riktning. Räknaren har även en indata port som kommer från
utdatan till port \mono{10}. Om port \mono{10} skrivs till med de två högsta
bitarna som "10" kommer räknaren laddas med de 6 lägre bitarna.

\mono{Y} består av en räknare på fem bitar och fungerar på liknande sätt som
räknaren för \mono{X}. \mono{Y} ska dock räkna olika högt beroende på vilket
läge som är valt. Vid 6-bitarsläge ska \mono{Y} loopa mellan 0-19 och vid
8-bitarsläge mellan 0-14. Här måste dataingången och \mono{CLR}-ingången
tillsammans med komparatorer användas för att åstadkomma loopning vid randerna.

Bildminnet är ett block-RAM med två 1-bitars dataportar, en för
läsning/skrivning av processorn och en för läsning av VGA-motorn. Bildminnet
går på systemklockan som ligger på \SI{100}{\mega\hertz}. VGA-motorn läser en
pixel i taget, den skickar rad och kolumn för en pixel och förväntar sig dess
färg samma klockintervall. Det är möjligt endast på grund av att VGA-motorn går
på \SI{25}{\mega\hertz}. Vid läsning ger bildminnet pixeln ett klockintervall
senare men eftersom bildminnet går på systemklockan hinner rätt värde nå
VGA-motorn i god tid.

Från processorns sida är tidsrestriktionen ännu mer avslappnad. Processorn går
som högst på \SI{14}{\mega\hertz} och intervallet mellan två läsningar eller
skrivningar kan som minst vara 10 T-states. Dock måste flera läsningar göras
eftersom antingen ska sex eller åtta bitar läsas. För läsning av processorn
läser LCD-kontrollern alltid från bildminnet, en bit i taget. När den kommer
till den sista biten börjar den om på den första. Bitarna lagras i en 8-bitars
buffer på rätt plats. Beroende på läget läses sex eller åtta bitar. Den här
buffern är dock inte registret som processorn läser ifrån. När processorn läser
förväntar den sig värdet där pekaren pekade på sist en läsning gjordes.
Processorn läser värdet från registret och i och med läsningen antar registret
det nuvarande värdet från den buffern.

Skrivning fungerar på liknande sätt. Då stannas läsningen och dataporten
används för skrivning istället. Kontrollern får in en page via port \mono{11}
och sätter igång en räknare. Bit för bit skrivs page:n till bildminnet. När
räknaren har nått bit 5 eller bit 7 är den färdig och dataporten överlåts åter
för läsning.
\end{document}
