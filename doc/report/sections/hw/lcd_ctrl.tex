\documentclass[main.tex]{subfiles}

\begin{document}
% TODO merga till nedan, bildminne ingår i lcd t6a04
\subsubsection{Bildminnet}
Bildminnet är den komponent som innehåller all data för den bild som ska ritas
ut. Det är 120x64 stort trots att monitorn endast är 96x64 stor. De extra
pixlarna används bara i vissa applikationer som utnyttjar ytan som en buffer
eller som extra minne. Det finns en funktion inbyggt i OS:et att flytta buffern
fram och tillbaka men det används inte aktivt av OS:et självt. Bildminnet
lagrar endast en bit per pixel eftersom bilden som ritas bara antingen har
påslagna eller avstängde pixlar.  När VGA-motorn skickar en rad(0-64) på 6
bitar och en kolumn(0-120) på 5 bitar in till bildminnet räknar bildminnet fram
till det värdet och skickar rätt data tillbaka till VGA-motorn. Bildminnet är
väldigt nära kopplat med LCD-kontrollern. I avsnittet om LCD-kontrollern står
det också mer exakt om hur de båda är sammankopplade.

\subsubsection{LCD-kontrollern}
LCD-kontrollern är den komponent som kopplar samman Bildminnet och processorn.

LCD-kontrollern har två olika modes. 6- och 8-bitars mode. Det väljer om porten
ska överföra 6 eller 8 bitar åt gången. När kontrollern befinner sig i 6-bit
mode kommer bildminnet vara uppdelat i 20 stycken pages innehållande 64 ord
vardera. När den är i 8-bit mode kommer de vara 15 pages också då med 64 ord.
LCD-kontrollern ser, med hjälp av det bit-modet, till att rätt data skrivs till
bildminnet.

I LCD-kontrollern finns ett antal räknare.  X som räknar upp/ned till 64 vilket
är radnummret i bildminnet. Skrivning till eller läsning från bildminnet gör
att X räknaren automatiskt räknar upp/ned.  Y som räknar upp/ned pages i
bildminnet. Den räknar antingen upp/ned till 15 eller 20 beroende på vilket
mode som LCD-kontrollern befinner sig i. Skrivning till eller läsning från
bildminnet gör att Y räknaren automatiskt räknar upp/ned.  Z som räknar upp/ned
till 64 och räknar fram startadressen på bildminnet. Det värde Z har kommer
vara start för bilden som ritas.
\end{document}
