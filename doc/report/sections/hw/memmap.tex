\documentclass[main.tex]{subfiles}

\begin{document}
\subsubsection{Minneskontroller}
% kanske flytta nedan till teori
Z80-processorn har en databuss på 16 bitar. Den kan därmed peka ut 65536
platser eller 64 KiB av minne. TI83p har däremot ett 512 KiB ROM och ett 32 KiB
RAM. För kunna komma åt allt har de 64KiB av det virtuella minnet delats upp i
fyra 16 KiB pages. Varje page refererar till en lika stor page i det fysiska
minnet. Dessa pages kan bytas ut för att komma åt en annan del av det fysiska
minnet.  ROM är uppdelat i 32 pages (\mono{ROM 00}\dots\mono{ROM 1F}) och RAM
är uppdelat i 2 pages (\mono{RAM 0} och \mono{RAM 1}).
\vspace{-4mm}
\begin{figure}[H]
    \begin{subfigure}{0.5\textwidth}
        \center
        \ttfamily\arraybackslash
        \small
        \begin{tabular}{r|m{3.5cm}|l}
            \multicolumn{1}{c}{\normalfont start} &
            \multicolumn{1}{c}{\normalfont page} &
            \multicolumn{1}{c}{\normalfont slut} \\ \cline{2-2}
            0000 & ROM 00    & 3FFF \\ \cline{2-2}
            4000 & MEMPAGE A & 7FFF \\ \cline{2-2}
            8000 & MEMPAGE B & BFFF \\ \cline{2-2}
            C000 & RAM 0     & FFFF \\ \cline{2-2}
        \end{tabular}
        \caption{läge 0}
    \end{subfigure}
    \begin{subfigure}{0.5\textwidth}
        \center
        \ttfamily\arraybackslash
        \begin{tabular}{r|m{3.5cm}|l}
            \multicolumn{1}{c}{\normalfont start} &
            \multicolumn{1}{c}{\normalfont page} &
            \multicolumn{1}{c}{\normalfont slut} \\ \cline{2-2}
            0000 & ROM 00           & 3FFF \\ \cline{2-2}
            4000 & MEMPAGE A (jämn) & 7FFF \\ \cline{2-2}
            8000 & MEMPAGE A        & BFFF \\ \cline{2-2}
            C000 & MEMPAGE B        & FFFF \\ \cline{2-2}
        \end{tabular}
        \caption{läge 1}
    \end{subfigure}
    \caption{Layout av virtuellt minne för de två lägena.}
    \label{fig:virtual}
\end{figure}
\vspace{-4mm}
TI83p använder två olika lägen som kan ses i figur \ref{fig:virtual}. Läget
bestäms av bit 0 i port \mono{04}. Page A och B kan referera till vilken page
som helst i minnet. De väljs med port \mono{06} respektive \mono{07}. Bit 6
väljer om det är en page från \mono{ROM} eller \mono{RAM}. De lägsta bitarna
bestämmer dess nummer. Om till exempel \mono{out (06h),a} exekveras då \mono{A}
är \mono{41} så sätts page A till \mono{RAM 1}.

TI83p använder sig av två separata minnen men vår implementation använder
endast ett stort kontinuerligt minne. ROM har lagts som de första 512 KiB och
RAM har lagts som de 32 KiB direkt efter ROM. \mono{RAM 1} ligger dock före
\mono{RAM 0}. Den fysiska placeringen kan ses i figur \ref{fig:physical}.

Minneskontrollerns uppgift är att översätta den virtuella adressen från
processorn till den fysiska adressen utefter läget samt page A, B och därefter
skicka den till det fysiska minnet. Implementationen fungerar genom att först
bestämma A och B utefter port \mono{06} och \mono{07} och därefter muxa de till
de fyra olika platserna utefter det nuvarande läget. Därefter muxas en av dessa
fyra pages utefter de högsta bitarna i den virtuella adressen. Sedan läggs
bitarna i page:n ihop med de låga bitarna från den virtuella adressen för att
få den exakta fysiska adressen.
\vspace{-4mm}
\begin{figure}[H]
    \center
    \small\ttfamily\arraybackslash
    \begin{tabular}{c|c|c}
        \multicolumn{1}{c}{\normalfont startadress} &
        \multicolumn{1}{c}{\normalfont page} &
        \multicolumn{1}{c}{\normalfont slutadress} \\ \cline{2-2}
        00000  & ROM 00  & 03FFF \\ \cline{2-2}
        04000  & ROM 01  & 07FFF \\
        \multicolumn{1}{c}{\vdots} &
        \multicolumn{1}{c}{\vdots} & \vdots \\
        78000  & ROM 1E  & 7BFFF \\ \cline{2-2}
        7C000  & ROM 1F  & 7FFFF \\ \cline{2-2}
        80000  & RAM 1   & 83FFF \\ \cline{2-2}
        84000  & RAM 0   & 07FFF \\ \cline{2-2}
        88000  & oanvänd & 8BFFF \\ \cline{2-2}
        \vdots & \vdots  & \vdots \\
    \end{tabular}
    \caption{Placering av pages i det fysiska minnet.}
    \label{fig:physical}
\end{figure}
\end{document}
