\documentclass[main.tex]{subfiles}

\begin{document}
\subsubsection{Minneskontroller}
\label{sec:mmap}
TI-83p använder sig av två separata minnen men i implementation används endast
ett stort kontinuerligt minne. ROM har lagts som de första 512 KiB och RAM har
lagts som de 32 KiB direkt efter ROM. \mono{RAM 1} ligger dock före \mono{RAM
0}. Den fysiska placeringen kan ses i figur \ref{fig:physical}.
\begin{figure}[H]
    \center
    \small\ttfamily\arraybackslash
    \begin{tabular}{c|c|c}
        \multicolumn{1}{c}{\normalfont startadress} &
        \multicolumn{1}{c}{\normalfont page} &
        \multicolumn{1}{c}{\normalfont slutadress} \\ \cline{2-2}
        00000  & ROM 00  & 03FFF \\ \cline{2-2}
        04000  & ROM 01  & 07FFF \\
        \multicolumn{1}{c}{\vdots} &
        \multicolumn{1}{c}{\vdots} & \vdots \\
        78000  & ROM 1E  & 7BFFF \\ \cline{2-2}
        7C000  & ROM 1F  & 7FFFF \\ \cline{2-2}
        80000  & RAM 1   & 83FFF \\ \cline{2-2}
        84000  & RAM 0   & 07FFF \\ \cline{2-2}
        88000  & oanvänd & 8BFFF \\ \cline{2-2}
        \vdots & \vdots  & \vdots \\
    \end{tabular}
    \caption{Placering av pages i det fysiska minnet.}
    \label{fig:physical}
\end{figure}
Minneskontrollerns uppgift är att översätta den virtuella adressen från
processorn till den fysiska adressen utefter läget samt page A, B och därefter
skicka den till det fysiska minnet. Implementationen fungerar genom att först
bestämma A och B utefter den nuvarande buffern till port \mono{06} och
\mono{07} och därefter muxa dem till de fyra olika platserna utefter det
nuvarande läget. Därefter muxas en av dessa fyra pages utefter de högsta
bitarna i den virtuella adressen. Sedan läggs bitarna i page:n ihop med de låga
bitarna från den virtuella adressen för att få den exakta fysiska adressen.
\end{document}
