\documentclass[main.tex]{subfiles}

\begin{document}
\subsubsection{Minnesgränssnitt}
Minnesgränssnittet ansvarar för att översätta TI-ASIC:ens läs- och
skrivsignaler till det externa minnets kontrollsignaler. Minneskontrollern i
TI-ASIC:en skickar endast en läs- eller skrivsignal till gränssnittet. Det
externa minnet har däremot fler kontrollsignaler som bland annat
\mono{chip\_enable}, \mono{output\_enable} och \mono{write\_enable}. Alla
kontrollsignaler för minnet kan ses i dess manual.\cite{m45}

\begin{figure}
    \begin{subfigure}{0.5\textwidth}
        \centering
        \ttfamily
        \begin{tabular}{r|c|}
            \cline{2-2}
            7C000 & AA \\ \cline{2-2}
            7C001 & BB \\ \cline{2-2}
            7C002 & CC \\ \cline{2-2}
            7C003 & DD \\ \cline{2-2}
            7C004 & EE \\ \cline{2-2}
            7C005 & FF \\ \cline{2-2}
        \end{tabular}
        \caption{Miniräknaren.}
    \end{subfigure}
    \begin{subfigure}{0.5\textwidth}
        \centering
        \ttfamily
        \begin{tabular}{r|c|c|}
            \cline{2-3}
            3E000 & BB & AA \\ \cline{2-3}
            3E001 & DD & CC \\ \cline{2-3}
            3E002 & FF & EE \\ \cline{2-3}
        \end{tabular}
        \caption{Det externa minnet.}
    \end{subfigure}
    \caption{Två olika synvinklar för samma block av minne.}
    \label{fig:mif}
\end{figure}

Gränssnittet utför även en annan viktig uppgift. Databussen till minnet är 16
bitar och minnesadressen pekar ut 16-bitars platser. Z80-processorn lagrar
förstås endast åtta bitar i taget eftersom dess databuss är åtta bitar. Ett
alternativ är att endast använda en av bytes:en i varje plats eftersom det
finns tillräckligt med utrymme. Ett problem med detta är att mjukvaran Adept
skriver och läser till båda bytes och det leder till att varannan byte missas
när processorn läser. Istället ser gränssnittet till att adressen halveras och
datan skrivs eller läses ifrån antingen övre eller undre delen av platsen i
minnet. Vid skrivning används \mono{upper\_byte}- och
\mono{lower\_byte}-signalerna för att förhindra att den andra byte:n i platsen
skrivs över. Figur \ref{fig:mif} visar de två olika uppläggen av minnet som
gränssnittet måste översätta emellan.

\end{document}
