\documentclass[main.tex]{subfiles}

\begin{document}
\newpage
\subsubsection{Registersektion}
Registersektionen består av tre komponenter; en registerfil, en
adressadderare och en adressinkrementerare. Blockschema \ref{diag:z80} visar
hur de är sammankopplade.

\paragraph{Registerfil}
Nästan alla register i processorn har implementerats med hjälp av ett litet
RAM. Flera register behöver sällan läsas eller skrivas till samtidigt så en
lässignal och en väg till databussen för varje register är onödigt. Som nämnt
tidigare kan ett par av två 8-bitars register användas som ett 16-bitars
register. Vid läsning och skriving till registerfilens RAM kan antingen 8-
eller 16-bitars kliv användas. För att referera till ett 8-bitars register
används ingången \mono{daddr} som är en adress på fem bitar. För att referera
till ett 16-bitars register används \mono{aaddr} som är en adress på fyra
bitar. \mono{aaddr} är identisk till \mono{daddr} förutom att den saknar bit 0
som avgör om det är den höga eller låga byte:n. Figur \ref{fig:rfext} visar
alla registers externa adresser, de adresser som instruktionsdekodaren
använder.

\begin{figure}[b!]
    \centering
    \begin{subfigure}{0.5\textwidth}
        \center
        \ttfamily
        \begin{tabular}{r|c|c|l}
            \multicolumn{1}{c}{daddr} &
            \multicolumn{1}{c}{hög} & \multicolumn{1}{c}{låg} &
            \multicolumn{1}{c}{daddr} \\ \cline{2-3}
            00 00000 & B   & C   & 00001 01 \\ \cline{2-3}
            02 00010 & D   & E   & 00011 03 \\ \cline{2-3}
            04 00100 & H   & L   & 00101 05 \\ \cline{2-3}
            06 00110 & A   & F   & 00111 07 \\ \cline{2-3}
            08 01000 & W   & Z   & 01001 09 \\ \cline{2-3}
            10 01010 & SPh & SPl & 01011 11 \\ \cline{2-3}
            12 01100 & IXh & IXl & 01101 13 \\ \cline{2-3}
            14 01110 & IYh & IYl & 01111 15 \\ \cline{2-3}
            16 10000 & I   & R   & 10001 17 \\ \cline{2-3}
            18 10010 & PCh & PCl & 10011 19 \\ \cline{2-3}
        \end{tabular}
        \caption{8-bitars kliv.}
        \label{fig:rfext8}
    \end{subfigure}%
    \begin{subfigure}{0.5\textwidth}
        \center
        \ttfamily
        \begin{tabular}{r|c|}
            \multicolumn{1}{c}{aaddr} &
            \multicolumn{1}{c}{register} \\ \cline{2-2}
            00 0000 & BC \\ \cline{2-2}
            01 0001 & DE \\ \cline{2-2}
            02 0010 & HL \\ \cline{2-2}
            03 0011 & AF \\ \cline{2-2}
            04 0100 & WZ \\ \cline{2-2}
            05 0101 & SP \\ \cline{2-2}
            06 0110 & IX \\ \cline{2-2}
            07 0111 & IY \\ \cline{2-2}
            08 1000 & IR \\ \cline{2-2}
            09 1001 & PC \\ \cline{2-2}
        \end{tabular}
        \caption{16-bitars kliv.}
    \end{subfigure}
    \caption{Registrernas externa adresser för registerfilen med 8- eller
    16-bitars {\it stride} eller kliv.}
    \label{fig:rfext}
\end{figure}

Dessa adresser har bland annat valts utefter kodningen i Z80:s instruktioner
för att förenkla instruktionsdekodaren. Instruktioner som refererar till
\mono{B}, \mono{C}, \mono{D}, \mono{E}, \mono{H}, \mono{L}, \mono{A},
\mono{BC}, \mono{DE}, \mono{HL} och \mono{AF} använder alltid samma kodning i
en del av instruktionen. De tre minst signifikanta bitarna för 8-bitars
register är tagna direkt från instruktionen och på liknande sätt de två sista
bitarna för 16-bitars register. Ett exempel är instruktionerna \mono{ld c,h}
och \mono{ld h,e} samt \mono{push bc} och \mono{push af}:
\begin{align*}
    &\mono{4C: ld c,h}& \quad &
    {\underbrace{\mono{0 1}}_\mono{ld}}\mono{ }
    {\underbrace{\mono{0 0 1}}_\mono{c}}\mono{ }
    {\underbrace{\mono{1 0 0}}_\mono{h}}
    \qquad
    &\mono{C5: push bc}& \quad &
    \mono{1 1 }
    {\underbrace{\mono{0 0}}_\mono{bc}}\mono{ }
    \mono{0 1 0 1}
    \\
    &\mono{63: ld h,e} &&
    {\underbrace{\mono{0 1}}_\mono{ld}}\mono{ }
    {\underbrace{\mono{1 0 0}}_\mono{h}}\mono{ }
    {\underbrace{\mono{0 1 1}}_\mono{e}}
    &\mono{F5: push af} &&
    \mono{1 1 }
    {\underbrace{\mono{1 1}}_\mono{af}}\mono{ }
    \mono{0 1 0 1}
\end{align*}

Det finns däremot undantag där instruktionens kodning har ändrats. Se
objektkoderna för instruktionerna \mono{ld b,(hl)}, \mono{ld b,a}, \mono{inc
(hl)} och \mono{inc a}:
\begin{align*}
    &\mono{47: ld b,(hl)}& \quad &
    {\underbrace{\mono{0 1}}_\mono{ld}}\mono{ }
    {\underbrace{\mono{0 0 0}}_\mono{b}}\mono{ }
    {\underbrace{\mono{1 1 0}}_\mono{a?}}
    \qquad
    &&\mono{34: inc (hl)} &\quad &
    \mono{0 0 }
    {\underbrace{\mono{1 1 0}}_\mono{a?}}\mono{ }
    \mono{1 0 0}
    \\
    &\mono{46: ld b,a} &&
    {\underbrace{\mono{0 1}}_\mono{ld}}\mono{ }
    {\underbrace{\mono{0 0 0}}_\mono{b}}\mono{ }
    {\underbrace{\mono{1 1 1}}_\mono{f?}}
    &&\mono{3C: inc a} &&
    \mono{0 0 }
    {\underbrace{\mono{1 1 1}}_\mono{f?}}\mono{ }
    \mono{1 0 0}
\end{align*}

Dessa instruktioner refererar inte till \mono{A} och \mono{F} som koderna
föreslår. Instruktioner som har adressen för vårt \mono{A} hänvisar egentligen
till \mono{(hl)}, platsen i minnet som \mono{HL} pekar på. Instruktioner som
använder adressen för vårt \mono{F} refererar egentligen till \mono{A}.
Anledningen till detta är att om instruktionernas kodning används kommer
\mono{AF} lagras som \mono{FA} istället. När \mono{AF} ska hanteras måste då
registerfilen byta plats på den höga och låga byte:n vid läsning och skrivning.
Lösningen till detta är att översätta \mono{111} till \mono{110} och \mono{110}
till \mono{111} direkt i instruktionsdekodaren. På så sätt kan \mono{A} och
\mono{F} enkelt hanteras som antingen två 8-bitars register eller ett 16-bitars
register. Notera att vid instruktioner som refererar till \mono{(hl)} eller i
vårt fall \mono{F} kommer värdet inte att hämtas från registerfilen och därmed
används inte adressen. Adressen till \mono{F} används endast vid \mono{push}
och \mono{pop} eftersom där måste en byte i taget överföras till eller från
stacken i minnet.

Ett alternativ för att implementera registerfilen för detta är att helt enkelt
skapa ett RAM med den struktur och de externa adresserna som i figur
\ref{fig:rfext8}. Då skrivs eller hämtas en byte från byte:n som \mono{daddr}
pekar på. Där \mono{aaddr} pekar på skrivs eller hämtas två bytes istället.

Det uppstår dock ett problem när \mono{ex} instruktionerna ska användas. Det
finns tre sådana instruktioner; \mono{ex af,af'}, \mono{ex de,hl} och
\mono{exx}. \mono{exx} växlar värdet på BC med BC', DE med DE' och även HL med
HL'. Dessa antar att det finns en kopia av vissa register som kan lagra
originalregistrets värde temporärt. Implementationen ser därför ut som i figur
\ref{fig:rfint} istället.

\begin{figure}[b]
    \center
    \ttfamily\arraybackslash
    \begin{tabular}{r|c|c|l}
        \multicolumn{1}{c}{adress} &
        \multicolumn{1}{c}{hög} & \multicolumn{1}{c}{låg} &
        \multicolumn{1}{c}{adress} \\ \cline{2-3}
        00 00000 & B   & C   & 00001 01 \\ \cline{2-3}
        02 00010 & B   & C   & 00011 03 \\ \cline{2-3}
        04 00100 & D   & E   & 00101 05 \\ \cline{2-3}
        06 00110 & D   & E   & 00111 07 \\ \cline{2-3}
        08 01000 & H   & L   & 01001 09 \\ \cline{2-3}
        10 01010 & H   & L   & 01011 11 \\ \cline{2-3}
        12 01100 & A   & F   & 01101 13 \\ \cline{2-3}
    \end{tabular}
    \begin{tabular}{r|c|c|l}
        \multicolumn{1}{c}{adress} &
        \multicolumn{1}{c}{hög} & \multicolumn{1}{c}{låg} &
        \multicolumn{1}{c}{adress} \\ \cline{2-3}
        14 01110 & A   & F   & 01111 15 \\ \cline{2-3}
        16 10000 & W   & Z   & 10001 17 \\ \cline{2-3}
        18 10010 & SPh & SPl & 10011 19 \\ \cline{2-3}
        20 10100 & IXh & IXl & 10011 21 \\ \cline{2-3}
        22 10110 & IYh & IYl & 10111 23 \\ \cline{2-3}
        24 11000 & I   & R   & 11001 25 \\ \cline{2-3}
        26 11010 & PCh & PCl & 11011 27 \\ \cline{2-3}
    \end{tabular}
    \caption{Registernas interna organisation i registerfilens RAM.}
    \label{fig:rfint}
\end{figure}

Notera att det inte finns ett primärregister och ett ``prim''-register utan
istället två identiska kopior av varje register. Detta är för att istället för
att flytta runt på värdena varje gång ett byte sker ändras endast pekaren
till det registret till den andra kopian.

Registerfilen måste därmed lagra pekare som avgör vilket register som ska
användas. Totalt behövs vippor för att hålla koll på alla växlingar:
\begin{labeling}{indentzz}
\item[\mono{reg}]
    Avgör om \mono{BC}, \mono{DE} och \mono{HL} ska hänvisas till den första
    eller andra kopian av registret. Detta motsvarar bit 1 i adressen. Den
    skiftas när \mono{rf\_swp} är
    \mono{reg}.
\item[\mono{af}]
    Avgör om \mono{AF} ska hänvisas till första eller andra kopian på liknande
    sätt.
\item[\mono{dehl0}]
    Avgör om \mono{DE} och \mono{HL} ska hänvisa till sig själva eller varandra
    när \mono{reg} är noll. Ändras endast då \mono{rf\_swp} är \mono{dehl} och
    \mono{reg} är noll. Om endast en vippa användes skulle även \mono{DE'} och
    \mono{HL'} byta plats från programmerarens sida.
\item[\mono{dehl1}]
    Som \mono{dehl0} fast då \mono{reg} är ett.
\item[\mono{afwz}]
    Avgör om \mono{AF} och \mono{WZ} ska hänvisa till sig själva eller
    varandra. Denna används endast internt så det är okej att den även ändrar
    \mono{AF'} eftersom den byter tillbaka innan instruktionen är avklarad.
\end{labeling}

Detta är implementerat med hjälp av ett kombinatoriskt nät som översätter
adressen till dess nuvarande position i registerfilens RAM utifrån vad
D-vipporna har för nuvarande värde.

\begin{figure}
    \center
    \includegraphics{rfbaddr.eps}
    \caption{Två identiska kombinatoriska nät används för att översätta
    adresserna \mono{baddr} och \mono{daddr} till de interna adresserna.}
    \label{fig:rfbaddr}
\end{figure}

\paragraph{Adressinkrementerare} % TODO
Mellan adressbussen och ingången till registerfilen sitter en inkrementerare
som kan antingen inkrementera, dekrementera eller behålla adressbussens nuvarande
värde. Med hjälp av denna kan en adress från registerfilen läggas på
adressbussen och sedan antingen inkrementeras eller dekrementeras och därefter
laddas till samma register. På detta vis stegas \mono{PC} och \mono{SP} upp
eller ner. Det går däremot inte att hänvisa till två olika adresser i
registret. Det finns även \mono{inc} och \mono{dec}-instruktioner för 16-bitars
register som denna används till.

\paragraph{Adressadderare} % TODO
Det finns även en adderare som tar adressen \mono{aaddr} pekar på och adderar
den med värdet på databussen och därefter skickar resultatet till databussen.
Detta används för relativa adresseringar som till exempel \mono{jr d} som
hoppar \mono{d} steg i programmet och \mono{cp (iy+d)} som jämför värdet
\mono{d} steg efter värdet som \mono{IY} pekar på i minnet med \mono{A}.

% TODO kontrollsektionen nämner regfilens funktioner kort när den beskriver
% kontrollordet men det kanske behöver mer info i denna sektion

\end{document}
