\documentclass[main.tex]{subfiles}

\begin{document}
\subsection{Register}
Registerna är strukturerade som ett blokram sammankopplat med en multiplexer.
Multiplexerns roll i symbiosen är att hantera vilket register som skall skrivas
till respektive läsas från. Varje register har en speciell kod som används för
att muxen ska hålla koll på vilket register som pekas på. Muxen väntar sedan på
en läs- eller skrivsignal för att avgöra vad som ska göras med registret som
pekas på. Kodningen som använts för registerna är baserad på Zilogs
originalkodning som beskrivs i användarmanualen \cite{userman}. Det som skiljer
kodningen som används från originalkodningen är att två extra siffror har lagts
till och att kodningen på vissa av registrerna har förändrats. Siffrorna längst
fram lades till eftersom det var fler register som skulle hållas koll på än
originalsiffrorna räckte för.

Kodningen är binär och består av fem siffror. De två mest signifikanta bitarna
används för att skilja på Zilogs originalkodning och de registerna som
egentligen inte var en del av kodningen som ändå skulle hållas koll på.

Som beskrivet i teori-delen så finns det både 8-bitars- och 16-bitarsregister i
strukturen. Strukturen ser ut som följande:

\begin{center}
    \begin{tabular}{ r|c|c|l }
        \multicolumn{1}{c}{adress} &
        \multicolumn{1}{c}{hög} & \multicolumn{1}{c}{låg} &
        \multicolumn{1}{c}{adress} \\ \cline{2-3}
        00 00000 & \mono{B} & \mono{C}            & 00001 01 \\ \cline{2-3}
        02 00010 & \mono{B} & \mono{C}            & 00011 03 \\ \cline{2-3}
        04 00100 & \mono{D} & \mono{E}            & 00101 05 \\ \cline{2-3}
        06 00110 & \mono{D} & \mono{E}            & 00111 07 \\ \cline{2-3}
        08 01000 & \mono{H} & \mono{L}            & 01001 09 \\ \cline{2-3}
        10 01010 & \mono{H} & \mono{L}            & 01011 11 \\ \cline{2-3}
        12 01100 & \mono{A} & \mono{F}            & 01101 13 \\ \cline{2-3}
        14 01110 & \mono{A} & \mono{F}            & 01111 15 \\ \cline{2-3}
        16 10000 & \mono{W} & \mono{Z}            & 10001 17 \\ \cline{2-3}
        18 10010 & \multicolumn{2}{c|}{\mono{SP}} & 10011 19 \\ \cline{2-3}
        20 10100 & \multicolumn{2}{c|}{\mono{IX}} & 10011 21 \\ \cline{2-3}
        22 10110 & \multicolumn{2}{c|}{\mono{IY}} & 10111 23 \\ \cline{2-3}
    \end{tabular}
\end{center}
 
Figuren visar att det även finns en dublett av varje register. Det är för att
en swap-funktion ska hanteras. Det swap-funktionen gör är att byta innehållet i
ett register med innehåller på dess ''prim''-variant, som är ett
likadant,internt register. Om man skapar två identiska register kan båda
dubletterna agera som både ''prim''-variant och original. Vilket som är vilket
urskiljs med hjälp av en pekare.

Varje register - förutom \mono{SP}, \mono{IX} och \mono{IY} som är
16-bitarsregister - är 8-bitarsregister och har olika användningsområden.
Register \mono{A} är indirekt kopplat till ALU:n och register \mono{F} håller
koll på flaggorna. Register \mono{W} och \mono{Z} är endast interna register
och kan användas för att lagra värden som ska användas senare. I register
\mono{C} lagrar man värdena för vilken port som ska användas i \mono{IN} och
\mono{OUT} funktioner. Utöver dessa undantag kan alla register användas
likgiltigt. 8-bitarsregisterna kan även användas som 16-bitarsregister ifall de
paras ihop med sin andra halva. Vilka register som hör ihop med varandra kan
man se i figuren ovan. Det finns några instruktioner som bestämmer vilka
register som skall användas, men i de flesta instruktioner har användaren
valfrihet.

Det finns vissa register som inte är med i tabellen ovan men som ändå finns i
archiketuren. De registerna som inte är tabellen är sådana som inte behöver
kodning. Ett exempel på ett sådant register är instruktionsregistret \mono{IR}.
Eftersom instruktionsregistret inte är ett allmänt register behöver det inte en
explicit kodning för att nås, och undviker därför att kodas. Ett annat register
som inte har en kodning är det temporära registret som används i ALUn.  Vissa
ALU instruktioner så som \mono{CP} (compare), \mono{ADD} (addition) och
\mono{SUB} (subtraction) kräver två operander. Det är då det temporära
registret \mono{TMP} kommer in i bilden. \mono{TMP} används som den andra
indatan till ALUn.

\end{document}
