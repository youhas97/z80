\documentclass[main.tex]{subfiles}

\begin{document}
\subsection{TI-ASIC}
TI-ASIC:en innehåller alla de komponenter som är del av TI:s miniräknare
förutom processorn. Kommunikation mellan processorn och TI-ASIC:ens komponenter
sker via portsystemet. Vid \mono{in} och \mono{out}-instruktioner muxas data
till eller från rätt port utifrån adressbussen. De olika portarna leder
därefter till en eller flera komponenter. Blockschema \ref{diag:ti} ger en
överblick av denna struktur samt vilka komponenter som hanterar in- och utdata
för varje port. Nästan alla portar för TI-83p har implementerats. De portar som
saknas hanterar länkporten, exekveringsmask samt skrivskydd för ROM.

\subfile{sections/hw/pio}
\subfile{sections/hw/memmap}
\subfile{sections/hw/interrupt}
\subfile{sections/hw/lcd_ctrl}
\end{document}
