\documentclass[main.tex]{subfiles}

\begin{document}
\subsection{Z80-processorn}
Z80:n kan delas upp i tre sektioner; kontroll, register och ALU. De är anslutna
med varandra via databussen, addressbussen och en del tunnlar som alla kan ses
i blockschema \ref{diag:z80}. Kontrollsektionen har även kontrollsignaler till
alla delar av processorn i form av ett långt kontrollord. Detta är
\mono{cw}-signalen i blockschemat. Alla signaler i kontrollordet finns med i
blockschemat förutom muxadresser till adress- och databussen samt alla
lässignaler för alla register.

\subfile{sections/hw/ctrl}
\subfile{sections/hw/rf}
\subfile{sections/hw/alu}

\subsubsection{Bussar}
Det finns inte så mycket att skriva om bussarna då det är bland de enklare
delarna av architekturen. Det finns totalt tre bussar - en kontrollbuss, en
adressbuss och en databuss. Kontrolbussen hanterar endast kontrollsignaler så
som exempelvis avbrottssignaler, lässignaler, skrivsignaler etc. Databussen är
en 8-bitarsbuss som hanterar data som ska skrivas till minnet samt läsas från
minnet. Adressbussen är en 16-bitarsbuss som adresserna till minnesplatser som
skall användas skickas på. För att lägga ut information på adress- eller
databussen används en mux för att bestämma vilken källa informationen kommer
från. Samma princip gäller för att bestämma vilken destination informationen
ska till.

\end{document}
