\documentclass[main.tex]{subfiles}

\begin{document}
\section{Introduktion}
Följande är en introduktin till rapporten om konstruktionen av en
TI83p-miniräknare med en Z80-processor. Rapporten kommer huvudsakligen bestå av
fyra delar; en beskrivning av apparaten, en kort teorilektion, en djupare
beskrivning av hårdvaran samt allmäna slutsatser angående projektet.
\subsection{Bakgrund}
Eftersom gruppen bestod av fyra personer som ville göra projektet tillsammans
så fanns det fler möjligheter när det gällde storleken på apparaten som skulle
byggas. Till slut bestämdes det att skapandet av en allmän Zilog Z80 processor
skulle ske. Tanken var att den skulle användas för att bygga en TI84p så att
dess operativsystem samt andra program anpassade för miniräknaren skulle kunna
köras. Under projektet visade det sig att TI83p var väldigt lik men var enklare
att implementera, så den byggdes och dess operativsystem laddades istället.
\subsection{Syfte}
Syftet med konstruktionen var, först och främst, inlärning av kunskap om hur en
processor egentligen fungerar och få en inblick i hur uppbyggnaden av en
processor ser ut. Målet var att göra en allmän processor som kunde användas som
Z80 originalprocessorn användes, d.v.s. som processor för bl.a. TI-miniräknare
och diverse arkadspel.
\subsection{Källor}
För att förstå hur Z80:n är uppbyggd och hur den fungerar har boken {\it
Programming the Z80} \cite{zaks-z80prog} av Rodnay Zaks varit till väldig stor
nytta. Särskilt kapitel två om processorns organisation. Flera artiklar av Ken
Shirriff såsom {\it How the Z80's registers are implemented}
\cite{shirriff-reg} har också varit till stor hjälp. Zilogs användarmanual
\cite{z80um} har använts väl som referens för hur instruktioner, avbrott och
flaggor ska fungera. Tidsdiagrammen har även använts som bas för vår
implementation. En tabell från ClrHome \cite{clrhome} har används mycket som
snabb referens för alla instruktioners objektkoder.

För att förstå hur TI-miniräknarna fungerar har sidor som {\it Z80 Heaven}
\cite{z80heaven} och {\it WikiTI} \cite{brandonw} använts. Sidorna om
miniräknarens portar på WikiTI har varit väldigt användbara för att få reda på
hur TI-ASIC:en ska fungera. Källkoden för emulatorer har också varit väldigt
användbara för detta syfte. Specifikt har koden till {\it z80e} \cite{z80e} och
{\it Tilem2} \cite{tilem2} granskats väl. Dessa emulatorer har även använts för
att jämföra resultatet med vår implementation. Tilem2 har en väldigt användbar
debugger som gjorde att vi kunde köra en fungerande version parallellt med vår
version instruktion för instruktion. Detta var oerhört hjälpsamt i slutet av
projektet för att hitta de sista felen och slutligen få operativsystemet att
fungera.

\clearpage

\end{document}
