\documentclass[main.tex]{subfiles}

\begin{document}
\section{Apparaten}
Det här avsnittet är en överblick av hur apparaten och de diverse komponenterna
hänger samman och hur styrningen av apparaten går till.
\subsection{Uppsättning}
[visa bild på setup] FPGA kortet kopplas till en dator med programmet Adept
[källa/länk] som används för att hantera minnet hos FPGA-kortet. Programmet
fanns inte på MUXens datorer så en extern bärbar dator användes för att
överföra datan. Tangenbordet är kopplat till FPGA-kortets USB port för att
kunna skicka indata till FPGA-kortet. VGA-kabeln går till en 640x480 VGA-skärm.
Då miniräknaren har en 96x64 skärm har varje pixel skalats upp med faktorn 6.

\subsection{Användarhandledning}
För att kunna köra ett av de program som laddas in måste de hamna på rätt
adresser i minnet. Adresserna väljs i Adept, en för TI83-romet och en för
användarens program. CPU:ns programräknare startar alltid på adress 8000 så ett
hopp laddas även in för att komma till användarminnet där programmet ligger.
Programmet som laddas in måste vara en binär fil för att FPGA:n ska kunna tolka
filen. Därför måste programmet först konverteras från assembly-kod till en
sådan fil. Konverteringen sker via z80asm som är en assembler för z80
instruktioner.

I det här bygget har användaren möjlighet att öka frekvensen till 15MHz, som är
frekvensen TI-84 använder, via en switch.


%--TODO
%-vad gör spakarna
%-vad menas med LED

\clearpage
\end{document}
