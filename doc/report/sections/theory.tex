\documentclass[main.tex]{subfiles}

\begin{document}
\section{Teori}
\subsection{Z80}
Z80 processorn var en revoltionerande processor på 70-talet då den var kapabel
att göra allt Intels 8080 kunde samt hade många fler instruktioner, fler
funktioner och register samt förenklade hårdvarukopplingar. Tanken med det här
projektet är att efterlikna en TI-83p miniräknare, som använder en Z80
processor.
\subsubsection{8-bitars register}
\begin{itemize}
    \item A, även kallat ''ackumulator'', är det primära registret för
    aritmetiska operationer och för att nå minnet. Registret är direkt kopplat
    till ALU:n och kan på så sätt användas som indata för alla matematiska
    operationer.
    \item B - Ofta använt som en 8-bitars räknare.
    \item C - Används när användaren vill läsa/skicka in/ut signaler. Registret
    används som ett gränssnitt för hårdvaruportar.
    \item D - Används i praktiken oftast konjunkt med register E och sällan som
    8-bitars register.
    \item E - Används, precis som D, sällan i 8-bitarsform och paras istället
    ihop med D.
    \item F - Registret som innehåller flaggorna. Användaren har ingen extern
    kontroll över F och registret kan inte manipuleras med hjälp av
    LD-instruktioner och dylikt. ALU-instruktioner sätter flaggorna genom att
    ändra värdet i F
    \item H - Precis som D så används sällan i 8-bitarsform och används
    istället parvis med L.
    \item L - Används oftast parvis med H.
    \item I - I används i samband med avbrottshantering. Lagrar den översta
    byten i interrupt mode 2.
    \item W - Ett internt register som används för lagring.
    \item Z - Ännu ett intert register som används för lagring.
\end{itemize}
\subsubsection{16-bitars register}
\begin{itemize}
    \item AF - Används sällan då F lagrar värdet på flaggorna.
    \item BC - Används av vissa instruktioner som en byte-räknare för att uppnå
    upprepningar i vissa instruktioner. Kan också användas som en 16-bitars
    räknare.
    \item DE - Lagrar adressen av en minnesplats, oftast en destination.
    \item HL - Ett generellt 16-bitars register. Om ett 16-bitars register ska
    användas så brukar HL vara förstahandsvalet. De vanligaste
    användningsområdena är lagring av 16-bitars aritmetik och adresser till
    saker så som strängar, bilder, etiketter etc.\\
    Anmärkning: HL lagrar ofta originaladressen, till skillnad från DE som
    oftast lagrar destinationen.
    \item PC - ''Program Counter'' som lagrar minnesplatsen som processorn tar
    instruktioner från. Inga funktioner kan förändra PC, förutom genom att
    faktiskt hoppa till en annan adress i minnet.
    \item SP - Stackpekaren, pekar på den nuvarande adressen till toppen av stacken.
    \item WZ - En konjunktion av W och Z. WZ används också för lagring.
\end{itemize}

Information om de olika registerna hittades i artikeln  \emph{The Registers And Memory}. \cite{regsandmem}

\subsection{TI-83p}
Miniräknaren TI-83p är en grafritande miniräknare tillverkad av Texas
Instruments. Miniräknaren använder en Z80 processor klockad till 6MHz, en 96x64
monokrom LCD-skärm och 4 AAA batterier. Då projektets uppsättning består av ett
FPGA-kort och en 640x480 skärm så har varje pixel skalats upp med faktorn 6. I
det här bygget har användaren möjlighet att öka frekvensen till 15MHz, som är
frekvensen TI-84 använder, via en switch.
\clearpage

\end{document}
