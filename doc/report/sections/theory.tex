\documentclass[main.tex]{subfiles}

\begin{document}
\section{Teori}
En del bekantskap med processorn och miniräknaren som har replikerats kan vara
till nytta. När implementationen senare beskrivs finns det då något av en
förväntan för hur konstruktionens olika komponenter borde funktionera. Vissa
aspekt av processorn och miniräknaren kommer därmed introduceras i den här
sektionen. Dessa aspekt kommer återkomma i hårdvarusektionen och förtydliga
varför vissa designbeslut har tagits.

\subsection{Processorn Z80}
Z80 är en processor från 1976 tillverkad av Zilog. Den är fullt kompatibel med
Intel 8080 men har även extra instruktioner och register. Mer detaljerad
information om processorn och information om andra aspekt än de som följer
nedan kan hittas i Zilog:s användarmanual.\cite{z80um}
\begin{figure}[H]
    \center
    \includegraphics[width=1.00\textwidth]{z80arch.eps}
    \caption{Ett approximativt blockschema av Z80-processorns interna
    organisation. Av {\it Appaloosa}, CC BY-SA 3.0.}
    \label{fig:z80arch}
\end{figure}

\subsubsection{Register}
Processorn har åtta 8-bitars register och tre 16-bitars register till godo för
programmerare. Register med en bokstav är åtta bitar och register med två
bokstäver är 16 bitar. Dessa kan till stor del användas för allmänna behov, men
vissa register har speciella instruktioner och specifika användningsområden.
Par av 8-bitars register kan även användas som 16-bitars register av vissa
instruktioner, som bland annat \mono{push BC} och \mono{add HL, DE}. Det finns
även ett registerpar som programmeraren inte har tillgång till. Dessa är
\mono{W} och \mono{Z} och används endast internt av processorn. Som
\mono{fig:z80arch} visar har varje 8-bitars register även en ``prim''-variant.
Dessa kan lagra undan registrernas värden temporärt vid till exempel avbrott.
\begin{labeling}{indent}
\item[\mono{A}]
    Kallas även för ``ackumulator'' och är det primära registret för
    aritmetiska operationer. Registret är direkt kopplat till ALU:ns ena
    operandregister och används ofta som in- och utdata för matematiska
    instruktioner, till exempel med instruktionen \mono{add a, b} som adderar
    \mono{A} med \mono{B} och lagrar resultatet i \mono{A}.
\item[\mono{F}]
    Registret som lagrar flaggorna. Vilken bit som representerar vad visas i
    figur \ref{fig:flags}. Under en ALU-instruktion laddas \mono{F} med
    flaggorna. Det finns inga flyttnings- eller räkneinstruktioner såsom
    \mono{ld} eller \mono{sub} som använder \mono{F}. \mono{F} ändras endast
    indirekt via ALU:n. Det är dock möjligt att läsa och skriva till \mono{F}
    via stacken om \mono{push af} och \mono{pop af} används tillsammans med
    andra push och pop-instruktioner.
\item[\mono{B}, \mono{C}]
    \mono{B} används av vissa instruktioner som räknare, till exemplet
    \mono{cpir}. \mono{C} bland annat för att lagra portnumret vid en
    \mono{in}- eller \mono{ut}-instruktion. Parvis används de två registerna
    som biträknare av vissa instruktioner.
\item[\mono{D}, \mono{E}]
    Används bland annat parvis för att lagra minnesadresser. Till exempel som
    destinationsadress för \mono{ldir} som flyttar ett block i minnet.
\item[\mono{H}, \mono{L}]
    Register som ofta parvis lagrar en minnesadress. Många instruktioner
    använder \mono{HL} för att referera till en plats i minnet. Dessa register
    kan växla värden med \mono{DE} genom \mono{ex de, hl}-instruktionen.
\item[\mono{IX}, \mono{IY}]
    Dessa används på liknande sätt som \mono{HL} men istället för att använda
    adressen som registret pekar på används relativ adressering. Ett exempel är
    instruktionen \mono{and (ix+d)} som utför \mono{and} på värdet som ligger
    \mono{d} platser efter adressen som \mono{IX} pekar på.
\item[\mono{SP}]
    Stackpekaren, lagrar adressen till den nuvarande toppen av stacken.
    Speciella instruktioner som bland annat \mono{push}, \mono{pop} och
    \mono{call} modifierar \mono{SP} medan programmerarens förmåga att direkt
    modifiera \mono{SP} är begränsad.
\item[\mono{I}]
    \mono{I} används i samband med avbrottshantering. Registret lagrar de
    översta åtta bitarna av adressen till en hoppadress för avbrottsläge 2.
\item[\mono{R}]
    Register för memory refresh. Z80:n har inbyggd refresh-hantering för
    dynamiska minnen.
\item[\mono{PC}]
    Instruktionspekaren, lagrar adressen till minnesplatsen som processorn ska
    hämta nästa instruktion ifrån. PC ändras med hoppinstruktioner som bland
    annat \mono{jp}, \mono{call} och \mono{rst}. Ändras även av processorn vid
    avbrott.
\end{labeling}

\subsubsection{Externa bussar}
\begin{wrapfigure}{r}{0.21\textwidth}
    \centering
    \vspace{-5mm}
    \includegraphics[width=0.21\textwidth]{z80pins.eps}
    \caption{Z80-processorns in- och utportar. Bild av Sakurambo, GFDL.}
    \label{fig:z80pins}
\end{wrapfigure}
Z80:n har en extern 8-bitars databuss och en extern 16-bitars adressbuss.
Processorn både skriver och läser till och från databussen men skriver endast
till adressbussen. Det finns också en så kallad kontrollbuss som består av
flera in- och utsignaler. Insignalerna består av \mono{INT}, \mono{NMI},
\mono{RESET}, \mono{BUSREQ} och \mono{WAIT}. \mono{INT} används för att
signalera ett avbrott, \mono{NMI} för en {\it nonmaskable interrupt}.
\mono{WAIT} används vid minnes- eller IO-operationer för att indikera att
indatan ännu inte är redo. \mono{BUSREQ} signalerar till processorn att en
enhet vill använda bussarna. Processorn ger då ut hög impedans till data- och
adressbussen. Utsignalerna består av \mono{HALT}, \mono{M1}, \mono{IORQ},
\mono{MREQ}, \mono{RD}, \mono{WR}, \mono{RFSH} och \mono{BUSAK}. \mono{HALT} är
aktiv när processorn är i halt-läge. \mono{M1} är aktiv när processorn är i
maskincykel ett. Om processorn hämtar från minnet när \mono{M1} är aktiv
innebär det att processorn försöker läsa en instruktion. Med hjälp av \mono{RD}
och \mono{WR} signalerar processorn att den ska läsa eller skriva. Samtidigt
signalerar processorn om den vill läsa/skriva till minnet eller till en
IO-enhet med hjälp av \mono{IORQ} och \mono{MREQ}-signalerna. \mono{IORQ}
tillsammans med \mono{M1} kan även indikera att processorn är redo att läsa
från databussen under ett avbrott. \mono{RFSH} används för att skicka
refresh-signaler till ett dynamiskt minne. \mono{BUSAK} går aktiv när
processorn har lagt hög impedans på bussarna efter en \mono{BUSREQ}-signal.

\subsubsection{Flaggor}
\label{sec:theory:flags}
\begin{figure}[b]
    \center
    \begin{tabular}{|l|c|c|c|c|c|c|c|c|}
        \hline
        bit     & 7 & 6 & 5 & 4 & 3 & 2 & 1 & 0 \\ \hline
        flagga  & S & Z & Y & H & X & P & N & C \\ \hline
    \end{tabular}
    \caption{Flaggornas uppsättning i \mono{F}-registret.}
    \label{fig:flags}
\end{figure}

\begin{labeling}{indentzzzzzzzzzzzz}
    \item[\mono{C}, carry]
        Används framförallt för att indikera minnessiffra vid addition och lån
        vid subtraktion. I dessa fall är \mono{C} en kopia av bit 8 från
        resultatet. För vissa rotationsinstruktioner antar \mono{C} istället en
        av operandens kanter.
    \item[\mono{N}, subtract]
        Indikerar att instruktionen är en subtraktion. Används därefter endast
        av \mono{daa} för att korrigera resultatet till BCD.
    \item[\mono{P}, parity/overflow]
        \mono{P} visar om en aritmetisk overflow har inträffat för aritmetiska
        instruktioner. För bitinstruktioner som bland annat \mono{xor} visar
        \mono{P} om resultatet har paritet (jämnt antal nollor). \mono{P}
        används även för att indikera om \mono{BC} är noll vid
        blockinstruktioner som \mono{ldir}. Flaggan kan även sättas till värdet
        av \mono{IFF2}, en av avbrottsvipporna. Det sker vid exekveringen av
        \mono{ld a, i} och \mono{ld a, r}.
    \item[\mono{X}]
        Odokumenterad och har ingen funktion, kallas även för \mono{f3}.  För
        de flesta instruktioner är \mono{X} en kopia av bit 3 från resultatet.
    \item[\mono{H}, half-carry]
        Indikerar carry för resultatet av de första fyra bitarna. Z80:n har en
        4-bitars ALU så den här flaggan kommer därför naturligt.
    \item[\mono{Y}]
        Som \mono{X} men kopia av bit 5.
    \item[\mono{S}, sign]
        Indikerar att resultatet är negativt om tolkat som ett
        tvåkomplementstal. Det är en kopia av bit 7 från resultatet.
    \item[\mono{Z}, zero]
        Indikerar att resultatet är noll.
\end{labeling}

\subsubsection{Input/Output}
Z80-processorn har ett system av portar för att hantera in- och utdata till
andra enheter utöver minne. När ett program vill skicka data till en enhet
används en av \mono{out}-instruktionerna. Då placeras portnumret på
adressbussens lägre åtta bitar och utdatan läggs på databussen av processorn så
att IO-enheten som korresponderar till den porten tar emot den. Under en
\mono{in}-instruktion placerar processorn porten på adressbussen men
korresponderande IO-enhet placerar sin data på databussen. För att det här
systemet ska fungera måste en dedikerad enhet utanför processorn hantera
muxning av data till och från rätt enhet utifrån porten på adressbussen.

\subsubsection{Avbrott}
Vid början av exekveringen av varje instruktion kontrollerar processorn om
\mono{INT} eller \mono{NMI} är aktiva. Vid en aktiv \mono{INT}-signal reagerar
processorn på olika sätt beroende på vilket läge som är inställt av
programmeraren. Processorn har tre olika avbrottslägen som kan väljas med
instruktionerna \mono{im 0}, \mono{im 1} och \mono{im 2}. Det finns även två
vippor \mono{IFF1} och \mono{IFF2}. Processorn svarar endast på en
\mono{INT}-signal om \mono{IFF1} är 1. Programmeraren kan välja dess värde med
instruktionerna \mono{ei} och \mono{di}. \mono{IFF2} används för att
tillfälligt lagra \mono{IFF1}:s värde under en {\it nonmaskable interrupt}.

Vid avbrott under läge 0 hämtar processorn en 8-bitars instruktion från
databussen och exekverar den. Instruktionen kan till exempel vara en \mono{rst
28h} som direkt hoppar till adress \mono{0x0028}. Som för alla avbrott lägger
processorn även \mono{PC} på stacken så att processorn kan återvända till
programmet när avbrottsrutinen är färdig. Avbrottsrutinen indikerar att den är
färdig med \mono{reti}-instruktionen. Under avbrottet återställs även
\mono{IFF1} så att ett nytt avbrott inte kan ske under ett pågående avbrott.
Avbrottsrutinen måste själv sätta igång avbrott igen med \mono{ei}. För att ett
avbrott inte ska kunna ske mellan \mono{ei} och \mono{reti} kommer processorn
inte acceptera avbrott förrän en instruktion efter att \mono{ei} har använts.

Vid läge 1 hoppar processorn alltid till adress \mono{0x0038}. Det här
motsvarar en \mono{rst 38h}-instruktion.

Vid läge 2 kommer processorn att hämta de lägre åtta bitarna av en adress från
databussen. I \mono{I}-registret ligger de åtta högre bitarna av adressen till
en hoppadress som ligger i minnet. Processorn kommer hämta hoppadressen från
denna adress i minnet och lägga den i PC så att ett hopp utförs.

Vid en \mono{NMI}-signal kommer processorn reagera som under läge 1 men den
hoppar till adress \mono{0x0066} istället för \mono{0x0038}. Dessa avbrott sker
oavsett om \mono{IFF1} är 1 eller 0. Vid sådana avbrott återställs \mono{IFF1}
precis som innan för att förhindra avbrott men \mono{IFF2} behåller nu värdet
av \mono{IFF1}. På så sätt kan det föregående värdet av \mono{IFF1} återställas
när avbrottsrutinen indikerar att den är färdig genom att exekvera
\mono{retn}-instruktionen.

\subsection{Miniräknaren TI-83 Plus}
\begin{wrapfigure}{r}{0.4\textwidth}
    \centering
    \includegraphics[width=0.4\textwidth, bb=0 0 284 578]{img/ti83p.jpg}
    \caption{En TI-83p-miniräknare från Texas Instruments. Bild av
    Westernelectric555, public domain.}
    \label{fig:ti83p}
\end{wrapfigure}
TI-83p är en grafritande miniräknare tillverkad av Texas Instruments.
Miniräknaren använder en Z80 processor klockad till \SI{6}{\mega\hertz}, en
$96\times64$ monokrom LCD-skärm. TI-miniräknarna har en inbyggd {\it
application specific integrated circuit} (ASIC) som bland annat muxar data
mellan processorn och portarna. TI-83p:an använder sig av en T6A04
LCD-kontroller. Detaljerad information om dess funktionalitet finns i dess
datablad\cite{t6a04}. Andra delar av miniräknarens ASIC är odokumenterade men
den innehåller bland annat hårdvarutimers, avbrottshanterare,
tangentbordskontroller, minnesmappning och minnesskydd.

TI-83p använder sig inte av alla Z80:ns funktioner. Bland annat används inte
{\it nonmaskable interrupts} eftersom ASIC:en aldrig skickar en
\mono{NMI}-signal. Det här innebär att IFF2 aldrig kommer skilja sig från IFF1.
{\it Bus request} används aldrig då \mono{BUSRQ} aldrig går aktiv.

Vid avbrott lägger ASIC:en ingen data på databussen vilket innebär att
avbrottssläge 0 inte går att använda, då instruktionen är obestämd.
Avbrottsläge 2 går att använda men eftersom databussen bestämmer de 8 sista
bitarna på adressen till hoppadressen måste ett helt block av 256 bytes fyllas
med hoppadressen. Miniräknarens operativsystem använder endast avbrottsläge 0.
Miniräknaren maskar avbrott från olika källor med hjälp av port \mono{03}. Vid
ett avbrott måste processorn stänga av avbrottet via port \mono{04} så att
\mono{int}-signalen inte fortfarande är hög när processorn åter igen aktiverar
avbrott.

\subsubsection{Portar}
För att ge en överblick över hur TI-83p kommunicerar med externa enheter följer
en kort beskrivning av varje port i miniräknaren.

\begin{labeling}{indentzzzzzzzzzzzzzzzzzzzz}
\item[\mono{00}: Länkport]
    TI-83p har en port som kan användas för att koppla ihop två miniräknare.
    För att kommunicera med den andra miniräknaren kan länkporten kontrolleras
    via port \mono{00}.
\item[\mono{01}: Tangentbord]
    Skrivning till port \mono{01} väljer vilka grupper av tangenter som ska
    registrera knapptryck. Läsning av porten ger en mask av alla nedtryckta
    knappar för de nuvarande valda grupperna.
\item[\mono{02}: Status]
    Läsning ger bland annat batterinivå och visar om ROM är skrivskyddat.
\item[\mono{03}: Avbrottsmask]
    Maska de fyra avbrottskällorna; nedtryckning av ON-knappen, den första
    hårdvarutimer:n, den andra timer:n och avbrott från länkporten. Läsning
    returnerar den nuvarande masken.
\item[\mono{04}: Minnesläge/avbrott]
    Vid läsning indikerar de första fyra bitarna vilket typ av avbrott som har
    skett. Bit 0 visar till exempel om ON-knappen har orsakat avbrott.
    Skrivning till porten bestämmer blanda annat vilket minnesläge som ska
    användas och frekvensen till miniräknarens hårdvarutimer:s.
\item[\mono{05}: Exekvering/länkdata]
    Läsning ger den byte som senast mottogs via länkporten. Skrivning väljer ut
    ROM pages som kan maskas med port \mono{16} för att förbjuda eller tillåta
    exekvering.
\item[\mono{06}: Page A]
    Välj ut page A. Bit 6 väljer om det är en page från \mono{ROM} eller
    \mono{RAM}. De lägsta bitarna bestämmer dess nummer. Om till exempel
    \mono{out (06h),a} exekveras då \mono{A} är \mono{41} sätts page A till
    \mono{RAM 1}. 
\item[\mono{07}: Page B]
    Som \mono{06} fast en andra page som kallas \mono{B}.
\item[\mono{10}: LCD-kontroll]
    Kontrollera LCD-kontrollern. Kan bland annat ställa in pekarens position
    och hur den ska förflyttas vid skrivning av data.
\item[\mono{11}: LCD-data]
    Skriv eller läs data från pekarens position i LCD:ns bildminne.
\item[\mono{14}: ROM-skrivskydd]
    Aktivera eller avaktivera modifiering av ROM.
\item[\mono{16}: Exekveringsmask]
    Välj vilka pages från port \mono{05} som ska tillåta exekvering.
\end{labeling}
De komponenter som har implementerats beskrivs i mer detalj nedan. Mer
detaljerad information om varje port kan även hittas på
WikiTI\cite{wikiti-ports}.

\subsubsection{Minne}
Z80-processorn har en adressbuss på 16 bitar. Den kan därmed peka ut 65536
platser, eller 64 KiB av minne eftersom varje plats är en byte. TI-83p har
däremot ett 512 KiB ROM och ett 32 KiB RAM. Det krävs härmed en separation
mellan processorns {\it virtuella} minne och miniräknarens {\it fysiska} minne,
eftersom det virtuella minnet inte kan kartläggas 1:1 till det fysiska minnet.
Istället har det virtuella minnet delats upp i fyra 16 KiB {\it pages}. Varje
page refererar till en lika stor page i det fysiska minnet. Eftersom ROM är 512
KiB är det uppdelat i 32 av dessa pages; \mono{ROM 00}\dots\mono{ROM 1F}. RAM
består av endast två pages; \mono{RAM 0} och \mono{RAM 1}. TI-83p har två olika
lägen som avgör vilka av dessa pages som virtuella minnets fyra pages ska
referera till. Dessa kan ses i figur \ref{fig:virtual}. I läge 0 pekar den
första page:n alltid på \mono{ROM 00}; den första page:n i ROM. Likaså pekar
den sista page:n alltid på den första RAM-page:n. De två andra kan dock
kontrolleras av programmeraren till att peka på vilken page som helst i minnet.
På så sätt kan processorn nå hela det fysiska minnet trots att dess adressbuss
endast är 16 bitar bred.
\begin{figure}
    \begin{subfigure}{0.5\textwidth}
        \center
        \ttfamily
        \begin{tabular}{r|m{3.5cm}|l}
            \multicolumn{1}{c}{\normalfont start} &
            \multicolumn{1}{c}{\normalfont page} &
            \multicolumn{1}{c}{\normalfont slut} \\ \cline{2-2}
            0000 & ROM 00    & 3FFF \\ \cline{2-2}
            4000 & MEMPAGE A & 7FFF \\ \cline{2-2}
            8000 & MEMPAGE B & BFFF \\ \cline{2-2}
            C000 & RAM 0     & FFFF \\ \cline{2-2}
        \end{tabular}
        \caption{läge 0}
    \end{subfigure}
    \begin{subfigure}{0.5\textwidth}
        \center
        \ttfamily
        \begin{tabular}{r|m{3.5cm}|l}
            \multicolumn{1}{c}{\normalfont start} &
            \multicolumn{1}{c}{\normalfont page} &
            \multicolumn{1}{c}{\normalfont slut} \\ \cline{2-2}
            0000 & ROM 00           & 3FFF \\ \cline{2-2}
            4000 & MEMPAGE A (jämn) & 7FFF \\ \cline{2-2}
            8000 & MEMPAGE A        & BFFF \\ \cline{2-2}
            C000 & MEMPAGE B        & FFFF \\ \cline{2-2}
        \end{tabular}
        \caption{läge 1}
    \end{subfigure}
    \caption{Layout av det virtuella minnet för de två lägena.}
    \label{fig:virtual}
\end{figure}

\subsubsection{Tangentbord}
\label{theory:kbd}
Tangentbordet har 50 knappar; 10 rader med 5 knappar vardera. Se figur
\ref{fig:ti83p}. ON-knappen har särskild funktionalitet; den skickar ett
avbrott till processorn vid nedtryckning. Den används bland annat för att väcka
processorn från halt-läge när miniräknaren är avstängd.

\begin{figure}[b]
    \center
    \resizebox{\textwidth}{!}{%
    \begin{tabular}{r|l|l|l|l|l|l|l|l|}
        \multicolumn{1}{c}{}
        & \multicolumn{1}{c}{bit 7} & \multicolumn{1}{c}{bit 6}
        & \multicolumn{1}{c}{bit 5} & \multicolumn{1}{c}{bit 4}
        & \multicolumn{1}{c}{bit 3} & \multicolumn{1}{c}{bit 2}
        & \multicolumn{1}{c}{bit 1} & \multicolumn{1}{c}{bit 0} \\\cline{2-9}
        Grupp 0 & & & & & UP & RIGHT & LEFT & DOWN
            \\\cline{2-9}
        Grupp 1 & & CLEAR & $\wedge$ & $\div$ & $\times$ & $-$ & $+$ & ENTER
            \\\cline{2-9}
        Grupp 2 & & VARS & TAN & ) & 9 & 6 & 3 & (-)
            \\\cline{2-9}
        Grupp 3 & STAT & PRGM & COS & ( & 8 & 5 & 2 & .
            \\\cline{2-9}
        Grupp 4 & X,T,$\Theta$,n & APPS & SIN & , & 7 & 4 & 1 & 0    
            \\\cline{2-9}
        Grupp 5 & ALPHA & MATH & $x^{-1}$ & $x^2$ & LOG & LN & STO &
            \\\cline{2-9}
        Grupp 6 & DEL & MODE & 2ND & Y= & WINDOW & ZOOM & TRACE & GRAPH
            \\\cline{2-9}
    \end{tabular}}
    \caption{Miniräknarens uppdelning av tangentbordets knappar i grupper.}
    \label{fig:keygroups}
\end{figure}

De resterande 49 knapparna har delats upp i 6 olika grupper som visas i figur
\ref{fig:keygroups}. Varje grupp består av upp till åtta knappar, vilket gör
att en grupp kan representeras av en byte där varje bit indikerar om en knapp
är nedtryckt eller inte. En nolla representerar en nedtryckt knapp.
Programmeraren kan aktivera grupper genom att skicka en byte till port
\mono{01}. En nolla på bit 0 aktiverar grupp 0, en nolla på bit 1 aktiverar
grupp 1 och så vidare upp till bit 6. Värdet \mono{1001 1111} aktiverar till
exempel grupp 5 och 6 och deaktiverar de resterande grupperna.

När programmeraren läser från port \mono{01} skickas en byte där alla
aktiverade grupper har AND:ats. Om till exempel de tidigarenämnda grupperna är
valda och användaren trycker ner tangenterna DOWN, LN och GRAPH kommer
$\mono{11111011} \cdot \mono{11111110} = \mono{11111010}$ tas emot.
DOWN-knappen registeras inte eftersom grupp 0 är avaktiverad. Programmeraren
vet inte om användaren tryckte på LN eller ZOOM eftersom de använder samma bit.
För att veta exakt vilka knappar som är nedtryckta måste en sökning göras genom
att gå igenom alla grupper med endast en grupp aktiverad i taget.

\subsubsection{LCD-kontroller}
TI-83p-miniräkaren använder en TOSHIBA T6A04 som LCD-kontroller. Kontrollern
har ett inbyggt RAM på 960 bytes som lagrar 64 rader med 120 pixlar vardera.
Eftersom skärmen är monokrom lagras endast en bit per pixel. LCD-skärmen visar
endast de första 96 kolumnerna i minnet, hela utrymmet kan dock användas för
lagring.

Minnet är uppdelat i {\it pages}, storleken på varje page kan vara antingen sex
eller åtta bitar. De två olika lägena visas i figur \ref{fig:lcdpages}. En
skrivning eller läsning sker alltid med en page i taget. Om läget är inställt
på sex bitar används endast de sex lägre bitarna av den data som skickas till
porten.

\begin{figure}[b]
    \centering
    \newcolumntype{e}{>{\center}m{8mm}}
\newcolumntype{s}{p{6mm}}

\begin{subfigure}{0.5\textwidth}
    \centering
    \begin{tabular}{c|e|e|c|e|e|}
        \multicolumn{1}{c}{} &
        \multicolumn{1}{c}{0} &
        \multicolumn{1}{c}{1} &
        \multicolumn{1}{c}{\dots} &
        \multicolumn{1}{c}{13} &
        \multicolumn{1}{c}{14} \\ \cline{2-3} \cline{5-6}
        0 &&&\dots&&\\ \cline{2-3} \cline{5-6}
        1 &&&\dots&&\\ \cline{2-3} \cline{5-6}
        \multicolumn{1}{c}{\vdots} &
        \multicolumn{1}{c}{\vdots} &
        \multicolumn{1}{c}{\vdots} &
        \multicolumn{1}{c}{} & 
        \multicolumn{1}{c}{\vdots} &
        \multicolumn{1}{c}{\vdots} \\ \cline{2-3} \cline{5-6}
        62 &&&\dots&&\\ \cline{2-3} \cline{5-6}
        63 &&&\dots&&\\ \cline{2-3} \cline{5-6}
    \end{tabular}
    \caption{8-bitars kliv, 15 pages per rad.}
\end{subfigure}%
\begin{subfigure}{0.5\textwidth}
    \centering
    \begin{tabular}{c|s|s|c|s|s|}
        \multicolumn{1}{c}{} &
        \multicolumn{1}{c}{0} &
        \multicolumn{1}{c}{1} &
        \multicolumn{1}{c}{\dots} &
        \multicolumn{1}{c}{18} &
        \multicolumn{1}{c}{19} \\ \cline{2-3} \cline{5-6}
        0 &&&\dots&&\\ \cline{2-3} \cline{5-6}
        1 &&&\dots&&\\ \cline{2-3} \cline{5-6}
        \multicolumn{1}{c}{\vdots} &
        \multicolumn{1}{c}{\vdots} &
        \multicolumn{1}{c}{\vdots} &
        \multicolumn{1}{c}{} & 
        \multicolumn{1}{c}{\vdots} &
        \multicolumn{1}{c}{\vdots} \\ \cline{2-3} \cline{5-6}
        62 &&&\dots&&\\ \cline{2-3} \cline{5-6}
        63 &&&\dots&&\\ \cline{2-3} \cline{5-6}
    \end{tabular}
    \caption{6-bitars kliv, 20 pages per rad.}
\end{subfigure}

    \caption{Uppdelning av pages för de två lägena.}
    \label{fig:lcdpages}
\end{figure}

Kontrollern har en inbyggd pekare som pekar på en page i bildminnet. Pekaren
består av ett \mono{X}-värde och ett \mono{Y}-värde. \mono{X} är vilken rad
page:n ligger på och \mono{Y} pekar ut kolumnen. Vid läsning kommer page:n som
pekaren refererar till skickas och vid skrivning kommer denna page skrivas över
i minnet. De olika lägena gör att pekaren kommer referera till olika platser i
minnet för lika värden på \mono{Y}. Om \mono{Y} exempelvis är 14 i 8-bitarsläge
kommer page:n längst till höger vara vald. Men om \mono{Y} är 14 i 6-bitarsläge
kommer den valda page:n ligga några pages in från höger kant.

Pekarens värde kan väljas manuellt av programmeraren med port \mono{11}. Men
vid varje läsning eller skrivning till dataporten \mono{11} kommer pekaren
antingen öka eller minska \mono{X} eller \mono{Y} ett steg beroende på pekarens
inställda riktning. Riktningen ställs också in med port \mono{10}.

LCD-kontrollern håller även koll på ett tredje värde; \mono{Z}. Det här värdet
avgör vilken rad \mono{X} ska börja räkna från vid läsning och skrivning. Om
\mono{X} och \mono{Y} exempelvis är noll och \mono{Z} är \mono{31} kommer den
första page:n på den mittersta raden (31) väljas. Eftersom \mono{X} börjar på
\mono{Z} kan värdet överstiga antalet av skärmens rader. Pekaren fortsätter då
räkna på första raden. Raden väljs därmed egentligen genom att beräkna
$\mono{X}+\mono{Z} \mod 64$.

\mono{Z} räknar inte upp eller ner vid läsning/skrivning utan sätts manuellt av
programmeraren via port \mono{10}. Syftet med \mono{Z} är att möjliggöra
vertikal scrollning utan att behöva flytta varje rad i minnet. Det här används
bland annat av operativsystemet för hemskärmen. Då ökar \mono{Z} med antalet
pixlar för höjden av en rad och den översta raden skrivs över med den nya raden
som då hamnar längst ner på skärmen eftersom pekaren går runt.

\subsection{Specifikationer}
Externa enheter har använts för att emulera miniräknarens LCD, tangentbord och
minne. För att kommunicera med dessa enheter har olika gränssnitt skapats, vars
implementation inte kommer beskrivas i detalj. Om detaljerad information om hur
dessa fungerar sökes förses hänvisningar till dokumentation för de olika
komponenterna nedan. VHDL-källkoden för gränssnittens implementationer är
förstås även tillgänglig för granskning.

\subsubsection*{VGA}
För att få ut en bild till VGA-skärmen har data skickats enligt VGA-standarden
med 8-bitars färg och upplösningen $640\times480$. Information och
tidsintervaller finns i manualen för Nexys3-kortet på sidan 15.\cite{nexys3}
\subsubsection*{PS/2}
För att ta emot knapptryck från ett tangentbord har PS/2-standarden använts vid
kommunikation. Hur standarden fungerar är beskrivet i Nexys3-manualen på sidan
13.\cite{nexys3}
\subsubsection*{Externt minne}
Ett 16MB RAM, modell Micron M45W8MW16 har använts för att emulera miniräknarens
ROM och RAM. Det externa minnets funktionalitet och gränssnitt kan hittas i
dess manual.\cite{m45} Implementationen använder asynkron läsning och
skrivning.

\clearpage
\end{document}
