\documentclass[main.tex]{subfiles}

\begin{document}
\section{Teori}
\subsection{Z80}
Z80 är en processor från 1976 tillverkad av Zilog. Den är fullt kompatibel med
Intel 8080 men har även extra instruktioner och register. Nedan följer en kort
beskrivning till vissa aspekt av processorn. Mer detaljerad information om
processorn kan hittas i Zilog:s användarmanual.
\cite{z80um}

\subsubsection{Register och flaggor}
\begin{itemize}
    \item A, även kallat ''ackumulator'', är det primära registret för
        aritmetiska operationer och för att nå minnet. Registret är direkt
        kopplat till ALU:n och kan på så sätt användas som indata för alla
        matematiska operationer.
    \item B - Ofta använt som en 8-bitars räknare.
    \item C - Används bland annat för att lagra lagra portnumret vid en in-
        eller utinstruktion.
    \item D, E - Används generellt parvis för att lagra minnesadresser.
    \item F - Registret som innehåller flaggorna. Användaren har ingen extern
        kontroll över F och registret kan inte manipuleras med hjälp av
        LD-instruktioner och dylikt. ALU-instruktioner sätter flaggorna genom
        att ändra värdet i F
    \item H,L - Register som ofta lagrar en minnesadress för kompatibibilitet
        med 8080. Många instruktioner använder HL för att referera till en
        plats i minnet.
    \item I - I används i samband med avbrottshantering. Lagrar den översta
        byten i interrupt mode 2.
    \item W, Z - Två interna register som används för lagring under
        instruktioner.
    \item PC - Lagrar minnesplatsen som processorn tar
        instruktioner från. Inga funktioner kan förändra PC, förutom genom att
        faktiskt hoppa till en annan adress i minnet.
    \item SP - Stackpekaren, pekar på den nuvarande adressen till toppen av
        stacken.
\end{itemize}
% TODO kort om flaggor, även odokumenterade

\subsubsection{Externa bussar}
Z80:n har en 8-bitars databuss och en 16-bitars adressbuss. Processorn både
skriver och läser till och från databussen men skriver endast till
adressbussen. Det finns också en så kallad kontrollbuss som består av flera in-
och utsignaler. Insignalerna består av \mono{INT}, \mono{NMI}, \mono{RESET},
\mono{BUSREQ} och \mono{WAIT}. \mono{INT} används för att signalera ett
avbrott, \mono{NMI} för ett {\it nonmaskable interrupt}. \mono{WAIT} används
vid minnes- eller IO-operationer för att indikera att indatan ännu inte är
redo. \mono{BUSREQ} signalerar till processorn att en enhet vill använda så att
processorn sätter databussen och adressbussen till högimpedans. Utsignalerna
består av \mono{HALT}, \mono{M1}, \mono{IORQ}, \mono{MREQ}, \mono{RD},
\mono{WR}, \mono{RFSH}, \mono{BUSACK}. \mono{HALT} är aktiv när processorn är i
halt-läge. \mono{M1} är aktiv när processorn är i maskincykel ett. Om
processorn hämtar från minnet när \mono{M1} är aktiv innebär det att processorn
hämtar en instruktion. Med hjälp av \mono{RD} och \mono{WR} signalerar
processorn att den ska läsa eller skriva. Samtidigt signalerar processorn om
den vill läsa/skriva till minnet eller till en IO-enhet med hjälp av
\mono{IORQ} och \mono{MREQ}-signalerna. \mono{RFSH} används för att skicka
refresh-signaler till ett dynamiskt minne. \mono{BUSACK} går aktiv när
processorn har lagt hög impedans på bussarna efter en \mono{BUSREQ}-signal.

\subsubsection{IO}
Z80-processorn har ett system av portar för att hantera in- och utdata till
andra enheter utöver minne. När ett program vill skicka data till en enhet
används en av \mono{OUT}-instruktionerna. Då placeras portnumret på
adressbussens lägre åtta bitar och utdatan läggs på databussen av processorn så
att IO-enheten som korresponderar till den porten tar emot den. Under en
\mono{IN}-instruktion placerar processorn porten på adressbussen men
korresponderande IO-enhet placerar sin data på databussen. För att det här
systemet ska fungera måste en dedikerad enhet utanför processorn hantera
muxning av data till och från rätt enhet utifrån porten på adressbussen.

\subsubsection{Avbrott}


\subsection{TI-83p}
Miniräknaren TI-83p är en grafritande miniräknare tillverkad av Texas
Instruments. Miniräknaren använder en Z80 processor klockad till 6MHz, en 96x64
monokrom LCD-skärm. TI-miniräknarna har en inbyggd {\it application specific
integrated circuit} (ASIC) som bland annat muxar data mellan processorn och
portarna. TI83p:an använder sig av en T6A04A LCD kontroller. Detaljerad
information om dess funktinalitet finns i dess datablad. Andra delar av
TI83p:ans ASIC är odokumenterade men den innehåller bland annat hårdvarutimers,
avbrottkontroller, tangentbordskontroller, minnesmappning och minnesskydd.

TI83p använder sig inte av alla Z80:ns funktioner. Bland annat används inte
{\it non-maskable interrupts} eftersom ASIC:en aldrig skickar en
\mono{NMI}-signal. Detta innebär även att IFF2 aldrig kommer skilja sig från
IFF1 vilket innebär att endast en d-vippa behövs. {\it Bus request} används
heller aldrig eftersom \mono{BUSRQ} aldrig går aktiv. Vid avbrott lägger
ASIC:en ingen data på databussen så \mono{IM 0} går inte att använda eftersom
då exekveras instruktionen på databussen. \mono{IM 2} går att använda men
eftersom databussen bestämmer de 8 sista bitarna på adressen till hoppadressen
måste ett helt block av 256 bytes fyllas med hoppadressen. Miniräknarens
operativsystem använder endast \mono{IM 0}.

\clearpage
\end{document}
