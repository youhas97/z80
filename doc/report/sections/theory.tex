\documentclass[main.tex]{subfiles}

\begin{document}
\section{Teori}
\subsection{Z80}
Z80 är en processor från 1976 tillverkad av Zilog. Den är fullt kompatibel med
Intel 8080 men har även extra instruktioner och register. Nedan följer en kort
beskrivning till vissa aspekt av processorn. Mer detaljerad information om
processorn kan hittas i Zilog:s användarmanual.\cite{z80um}

\subsubsection{Register}
    Processorn har åtta 8-bitars register och och tre 16-bitars register.
    Programmerare kan direkt komma åt och använda de flesta av dessa generellt.
    Vissa register har dock speciella instruktioner och specifika
    användningsområden. Par av 8-bitars register kan även användas som
    16-bitars register av vissa instruktioner som bland annat \mono{push BC},
    \mono{add HL, DE}. Register med en bokstav är 8 bitar och register med två
    bokstäver är 16 bitar.
\begin{itemize}
    \item \mono{A}, även kallat ``ackumulator'' är det primära registret för
        aritmetiska operationer och för att nå minnet. Registret är direkt
        kopplat till ALU:n och kan på så sätt användas som indata för
        matematiska operationer som till exempel \mono{add a, b}.
    \item \mono{B} - Vissa instruktioner använder \mono{B} som räknare.
    \item \mono{C} - Används bland annat för att lagra lagra portnumret vid en
        \mono{in}- eller \mono{ut}-instruktion. Tillsammans med B kan
    \item \mono{D}, \mono{E} - Används bland annat parvis för att lagra
        minnesadresser. Till exempel som destinationsadress för \mono{ldir}
        som flyttar ett block i minnet.
    \item \mono{F} - Registret som innehåller flaggorna. Under en
        ALU-instruktion laddas \mono{F} med flaggorna. Har inga flyttnings-
        eller räkneinstruktioner såsom \mono{ld} eller \mono{sub}. Det går dock
        att läsa och skriva till \mono{F} via stacken med \mono{push af} och
        \mono{pop af}.
    \item \mono{H}, \mono{L} - Register som ofta lagrar en minnesadress. Många instruktioner
        använder HL för att referera till en plats i minnet. Dessa register kan
        även byta plats med \mono{DE} med \mono{ex de, hl}-instruktionen.
    \item \mono{IX}, \mono{IY} - Dessa används på liknande sätt som \mono{HL}
        men istället för att använda adressen som registret pekar på används
        relativ adressering. Ett exempel är instruktionen \mono{and (ix+d)} som
        utför \mono{and} på värdet som ligger \mono{d} platser efter adressen
        som \mono{IX} pekar på.
    \item \mono{SP} - Stackpekaren, pekar på den nuvarande adressen till toppen av
        stacken. Speciella instruktioner som \mono{push}, \mono{pop} och
        \mono{call} modifierar \mono{SP} medan programmerarens förmåga att
        direkt modifiera \mono{SP} är begränsad.
    \item \mono{I} - \mono{I} används i samband med avbrottshantering. Lagrar
        den översta byten av adressen till en hoppadress för interrupt mode
        2.
    \item \mono{R} - Register för memory refresh. Z80:n har inbyggd
        refresh-hantering för dynamiska minnen.
    \item \mono{PC} - Lagrar minnesplatsen som processorn tar instruktioner
        från. Inga funktioner kan ändra PC direkt, endast indirekt med
        hoppinstruktioner.
        % jp **, jp (hl) fungerar precis som ld pc,**; ld pc,hl skulle göra.
        % så man kan ju typ ändra pc direkt, kanske omformulera/förtydliga
\end{itemize}

\subsubsection{Flaggor}
\begin{figure}[H]
    \center
    \begin{tabular}{|l|c|c|c|c|c|c|c|c|}
        \hline
        bit     & 7 & 6 & 5 & 4 & 3 & 2 & 1 & 0 \\ \hline
        flagga  & S & Z & Y & H & X & P & N & C \\ \hline
    \end{tabular}
    \caption{Flaggornas uppsättning i \mono{F}-registret.}
\end{figure}
\begin{itemize}
    \item \mono{C} carry - Används framförallt för att indikera minnessiffra
        vid addition och lån vid subtraktion. Vid aritmetisk operation är det
        en kopia av den bit 8. För vissa rotationsinstruktioner lagras en av
        operandens kanter i carry-flaggan.
        tvåkomplementstal. Det är en kopia av bit 7.
        kopia av bit 5 från resultatet.
    \item \mono{N} subtract - Indikerar att instruktionen är en subtraktion.
        Används endast av \mono{daa} för att korrigera resultatet till BCD.
    \item \mono{P} parity/overflow - Visar \mono{P} om en aritmetisk overflow
        har inträffat för aritmetiska instruktioner. För bitinstruktioner som
        bland annat \mono{xor} visar flaggan om resultatet har paritet (jämnt
        antal nollor). \mono{P} används även för att indikera om \mono{BC} är
        noll vid blockinstruktioner. Flaggan sätts även till värdet av
        \mono{IFF} av instruktionerna \mono{ld a, i} och \mono{ld a, r}.
    \item \mono{X} - Odokumenterad, har ingen funktion, kallas även för
        \mono{f3}. Generellt så är det en kopia av bit 3 av resultatet.
    \item \mono{H} half-carry - Indikerar carry för resultatet av de första
        fyra bitarna. Z80:n har en 4-bitars ALU så den här flaggan kommer
        naturligt.
    \item \mono{Y} - Odokumenterad, som \mono{X} men kopia av bit 5.
    \item \mono{S} sign - Indikerar om resultatet är negativt om tolkat som ett
    \item \mono{Z} zero - Indikerar att resultatet är noll.
\end{itemize}

\subsubsection{Externa bussar}
Z80:n har en 8-bitars databuss och en 16-bitars adressbuss. Processorn både
skriver och läser till och från databussen men skriver endast till
adressbussen. Det finns också en så kallad kontrollbuss som består av flera in-
och utsignaler. Insignalerna består av \mono{INT}, \mono{NMI}, \mono{RESET},
\mono{BUSREQ} och \mono{WAIT}. \mono{INT} används för att signalera ett
avbrott, \mono{NMI} för ett {\it nonmaskable interrupt}. \mono{WAIT} används
vid minnes- eller IO-operationer för att indikera att indatan ännu inte är
redo. \mono{BUSREQ} signalerar till processorn att en enhet vill använda så att
processorn sätter databussen och adressbussen till högimpedans. Utsignalerna
består av \mono{HALT}, \mono{M1}, \mono{IORQ}, \mono{MREQ}, \mono{RD},
\mono{WR}, \mono{RFSH}, \mono{BUSACK}. \mono{HALT} är aktiv när processorn är i
halt-läge. \mono{M1} är aktiv när processorn är i maskincykel ett. Om
processorn hämtar från minnet när \mono{M1} är aktiv innebär det att processorn
hämtar en instruktion. Med hjälp av \mono{RD} och \mono{WR} signalerar
processorn att den ska läsa eller skriva. Samtidigt signalerar processorn om
den vill läsa/skriva till minnet eller till en IO-enhet med hjälp av
\mono{IORQ} och \mono{MREQ}-signalerna. \mono{IORQ} med \mono{M1} kan även
indikera att processorn är redo att läsa från databussen under ett avbrott.
\mono{RFSH} används för att skicka refresh-signaler till ett dynamiskt minne.
\mono{BUSACK} går aktiv när processorn har lagt hög impedans på bussarna efter
en \mono{BUSREQ}-signal.

\subsubsection{IO}
Z80-processorn har ett system av portar för att hantera in- och utdata till
andra enheter utöver minne. När ett program vill skicka data till en enhet
används en av \mono{OUT}-instruktionerna. Då placeras portnumret på
adressbussens lägre åtta bitar och utdatan läggs på databussen av processorn så
att IO-enheten som korresponderar till den porten tar emot den. Under en
\mono{IN}-instruktion placerar processorn porten på adressbussen men
korresponderande IO-enhet placerar sin data på databussen. För att det här
systemet ska fungera måste en dedikerad enhet utanför processorn hantera
muxning av data till och från rätt enhet utifrån porten på adressbussen.

\subsubsection{Avbrott}
Vid början av exekveringen av varje instruktion kontrollerar processorn om
\mono{INT} eller \mono{NMI} är aktiva. Vid en aktiv \mono{INT}-signal reagerar
processorn på olika sätt beroende på vilket läge som är inställt av
programmeraren. Processorn har tre olika avbrottslägen som kan väljas med
instruktionerna \mono{im 0}, \mono{im 1} och \mono{im 2}. Det finns även två
D-vippor \mono{IFF1} och \mono{IFF2}. Processorn svarar endast på en
\mono{INT}-signal om \mono{IFF1} är 1. Programmeraren kan välja dess värde med
instruktionerna \mono{ei} och \mono{di}. \mono{IFF2} används för att lagra
\mono{IFF1}:s värde under en {\it nonmaskable interrupt}.

Vid avbrott under läge 0 hämtar processorn en 8-bitars instruktion från
databussen och exekverar den. Instruktionen kan till exempel vara en \mono{rst
28h} som direkt hoppar till adress 0x28. Processorn lägger även \mono{PC} på
stacken så att processorn kan återvända till programmet när avbrottsrutinen är
färdig och kör \mono{reti}. Under avbrottet återställs även \mono{IFF1} och
\mono{IFF2} så att ett nytt avbrott inte kan ske under avbrottsrutinen.
Avbrottsrutinen måste själv sätta igång avbrott igen med \mono{ei}. För att ett
avbrott inte ska kunna ske mellan \mono{ei} och \mono{reti}-instruktionerna så
kommer processorn inte acceptera avbrott förrän en instruktion efter att
\mono{ei} har använts.

Vid läge 1 hoppar processorn alltid till adress 0x38. Detta motsvarar en
\mono{rst 38h}-instruktion.

Vid läge 2 kommer processorn att hämta de lägre åtta bitarna av en adress från
databussen. I \mono{I}-registret ligger de åtta högre bitarna av adressen till
en hoppadress som ligger i minnet. Processorn kommer hämta hoppadressen från
denna adress minnet och lägga den i PC så att ett hopp utförs.

Vid en \mono{NMI}-signal kommer processorn reagera precis som under läge 1 men
att den hoppar till adress 0x66 istället för 0x38. Dessa avbrott sker oavsett
om \mono{IFF1} är 1 eller 0. Vid sådana avbrott återställs \mono{IFF1} precis
som innan för att förhindra avbrott men \mono{IFF2} antar nu värdet av
\mono{IFF1}. På så sätt kan föregående värdet av \mono{IFF1} återställas när
avbrottsrutinen är färdig och exekverar \mono{retn}-instruktionen.

\subsection{TI-83p}
Miniräknaren TI-83p är en grafritande miniräknare tillverkad av Texas
Instruments. Miniräknaren använder en Z80 processor klockad till 6MHz, en 96x64
monokrom LCD-skärm. TI-miniräknarna har en inbyggd {\it application specific
integrated circuit} (ASIC) som bland annat muxar data mellan processorn och
portarna. TI83p:an använder sig av en T6A04 LCD kontroller. Detaljerad
information om dess funktionalitet finns i dess datablad\cite{t6a04}. Andra
delar av TI83p:ans ASIC är odokumenterade men den innehåller bland annat
hårdvarutimers, avbrottshanterare, tangentbordskontroller, minnesmappning och
minnesskydd.

TI83p använder sig inte av alla Z80:ns funktioner. Bland annat används inte
{\it nonmaskable interrupts} eftersom ASIC:en aldrig skickar en
\mono{NMI}-signal. Detta innebär även att IFF2 aldrig kommer skilja sig från
IFF1 så endast en D-vippa behövs. {\it Bus request} används aldrig då
\mono{BUSRQ} aldrig går aktiv. Vid avbrott lägger ASIC:en ingen data på
databussen så \mono{IM 0} går inte att använda eftersom instruktionen är
obestämd. \mono{IM 2} går att använda men eftersom databussen bestämmer de 8
sista bitarna på adressen till hoppadressen måste ett helt block av 256 bytes
fyllas med hoppadressen. Miniräknarens operativsystem använder endast \mono{IM
0}.

% flytta memory mapping från hw?

\cite{m45}

\subsection{Specifikationer}
% TODO PS2 kbd, vga protocol
% TODO m45 datasheet
\clearpage
\end{document}
