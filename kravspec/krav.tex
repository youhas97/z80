\documentclass[12pt]{article}

\usepackage[utf8]{inputenc}
\usepackage[swedish]{babel}

\usepackage{tikz}
\usepackage{hyperref}
\usetikzlibrary{shapes,arrows}
\usepackage{fancyhdr}
\usepackage{float}

\usepackage[a4paper, total={14cm,20cm}]{geometry}
\lhead{TSEA83}
\rhead{VT18}
\pagestyle{fancy}


\date{\today}

\begin{document}
\begin{center}
\vspace*{2cm}
{\Huge Kravspecifikation: FPGA Z80}\\
\vspace{2cm}
{\large 16 februari 2018}\\
\vspace{2cm}
Noah Hellman: noahe116 \\
Dennis Derecichei: dende301 \\
Yousef Hashem: youha847 \\
Jakob Arvidsson: jakar180 \\
\end{center}

\vspace{2cm}
\tableofcontents
\newpage

\section{Bakgrund}
Z80 är en processor tillverkad av Zilog 1976 för inbyggda system. Den används
bland annat för TI84-miniräknare än idag. Processorn kom i början av pipelines
utveckling och har en fetch-exec cykel, med andra ord överlappar exekveringen
med hämtningen av nästa instruktion.

\section{Skiss}
\subsection{Arkitektur}
Z80 har en fetch-exec cycle. Vi behöver  hämta en instruktion på  1 cp samt
exekvera en instruktion på 1 cp. Vi behöver en instruktionsdekodare som
exekverar varje instruktion. Den kanske får en lång kritisk väg men det är okej
eftersom vi har en så pass låg frekvens. 

\subsection{TI84 kompatibilitet}
För att kunna köra TI84:s OS måste vi implementera varenda instruktion som
det använder korrekt. Vi måste dessutom  se till alla externa saker som
portar, tangentbord, skärm antingen fungerar eller inte orsakar någon krasch.

Vi har tänkt att skala upp TI84:s 96x64 skärm till VGA-skärmens 640x480
upplösning. Skärmen är gråskala så vi kan lagra varje pixel som en bit (ca 800
bytes) i ett videominne. Vi kan åtminstone inte enbart använda tiles eftersom
räknaren kan rita godtyckliga grafer. Tangentbordet kommer behöva mappas till
TI84:s knappar.

\section{Kravlista}
\subsection{Börkrav}
    Vi hoppas kunna exekvera all maskinkod skriven för TI84 så att vi kan köra
    dessa program med vår VGA-skärm som display och vårt USB-tangentbord som
    miniräknartangentbord. Vi vill därmed också kunna köra hela
    operativsystemet för TI84 och kunna använda det till fullo (köra TI-basic
    program, rita grafer, göra uträkningar, plotter etc).

\subsection{Skallkrav}
    Man ska kunna exekvera maskinkod kompilerad för en z80 med vår processor.
    Vi kanske inser att implementera varenda instruktion samt att få det
    kompatibelt med TI84-program blir för mycket arbete. Vi kan då se till att
    de mest nödvändiga instruktionerna fungerarar och skriva ett
    spel med hjälp av dem, exempelvis Pacman eller Space Invaders.

\section{Referenser}
\begin{itemize}
    \item Allmän information om Z80: \url{http://www.z80.info/}
    \item Information om Z80 och TI84: \url{http://z80-heaven.wikidot.com/}
\end{itemize}
\end{document}
