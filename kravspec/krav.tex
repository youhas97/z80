\documentclass[12pt]{article}

% encoding
\usepackage[utf8]{inputenc}
\usepackage[swedish]{babel}

% bibliography
\usepackage[backend=bibtex8]{biblatex}
\addbibresource{references.bib}
\usepackage{csquotes}

% equations
\usepackage{amsmath}

% split files
\usepackage{subfiles}

% table util, multicolumn etc
\usepackage{array}
\newcolumntype{C}[1]{>{\centering\arraybackslash}p{#1}}

% place figures here (H)
\usepackage{float}

% subfigures with captions
\usepackage{subcaption}

% titleformat for paragraphs
\usepackage{titlesec}

% hyperlinks for references, urls
\usepackage{hyperref}

% page dimensions
\usepackage[a4paper, total={16cm, 25cm}]{geometry}

% landscape appendices
\usepackage[paper=portrait,pagesize]{typearea}

% labeling
\usepackage{scrextend}

% include graphics
\usepackage{graphicx}

% wrap figures in text
\usepackage{wrapfig}

% captions on side of figure
\usepackage{sidecap}

% command listings
\usepackage{listings}
\lstset{basicstyle=\footnotesize\ttfamily}

% vhdl listings
\usepackage[newfloat]{minted}
\usepackage{caption}

% use colors
\usepackage{xcolor}

% SI
\usepackage{siunitx}

% diagrams
\usepackage{tikz}
\usepackage{tikz-timing}
% make arrows larger, IDA missing this package
%\usetikzlibrary{arrows.meta}
\usetikzlibrary{arrows,shapes,automata,petri,positioning,calc}

% monospace font
\newcommand\mono[1]{\texttt{#1}}

% header for sections under subsubsections
\titleformat{\paragraph}
{\normalfont\normalsize\bfseries}{\theparagraph}{1em}{}
\titlespacing*{\paragraph}
{0pt}{3.25ex plus 1ex minus .2ex}{1.5ex plus .2ex}


\date{\today}

\begin{document}
\begin{center}
\vspace*{2cm}
{\Huge Kravspecifikation: FPGA Z80}\\
\vspace{2cm}
{\large 16 februari 2018}\\
\vspace{2cm}
Noah Hellman: noahe116 \\
Dennis Dericichei: dende301 \\
Yousef Hashem: youha847 \\
Jakob Arvidsson: jakar180 \\
\end{center}

\newpage

\section{Bakgrund}
Z80 är en processor tillverkad av Zilog 1976 för inbyggda system. Den används
bland annat för TI84-miniröknare än idag. Processorn kom i början av pipelines
utveckling och har en fetch-exec cykel, med andra ord överlappar exekveringen
med hämtningen av nästa instruktion.

\section{Skiss}
\subsection{Arkitektur}
Z80 har en fetch-exec cycle. Vi behöver  hämta en instruktion på  1 cp samt
exekvera en instruktion på 1 cp. Vi behöver en instruktionsdekodare som
exekverar varje instruktion. Den kanske får en lång kritisk väg men det är okej
eftersom vi har en så pass låg frekvens. 

\subsection{TI84 kompatibilitet}
För att kunna köra TI84:s OS måste vi implementera varenda instruktion som
det använder korrekt. Vi måste dessutom  se till alla externa saker som
portar, tangentbord, skärm fungerar eller inte orsakar någon krasch.

Vi har tänkt att skala upp TI84:s 96x64 skärm till VGA-skärmens 640x480
upplösning. Skärmen är gråskala så vi kan lagra varje pixel som en bit (ca 800
bytes) i ett videominne. Tangentbordet kommer behöva mappas till TI84:s
knappar.

\section{Kravlista}
\subsection{Krav}
\begin{itemize}
    \item Fetch-exec cycle.
    \item Använda z80:s instruktioner.
    \item Skriva till VGA och visa på skärm.
    \item Ladda z80 asm-program via USB, vissa som andra har skrivit ska
        fungera.
\end{itemize}

\subsection{Bör}
\begin{itemize}
    \item Köra  TI84:s operativsystem.
    \item Göra  TI84 kompatibelt med VGA-skärm.
    \item Göra  TI84 kompatibelt med PC-tangentbord.
\end{itemize}

\section{Referenser}
\begin{itemize}
    \item Allmän information om Z80: \url{http://www.z80.info/}
    \item Information om Z80 och TI84: \url{http://z80-heaven.wikidot.com/}
\end{itemize}
\end{document}
